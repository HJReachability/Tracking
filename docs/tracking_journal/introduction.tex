% !TEX root = tracking.tex
\section{Introduction}
 As unmanned aerial vehicles (UAVs) and other autonomous systems become more commonplace, it is essential that they be able to plan safe motion paths through crowded environments in real-time. This is particularly crucial for navigating through environments that are \textit{a priori} unknown, because re-planning based on updated information about the environment is often necessary. 
 However, for many common dynamical systems, accurate and robust path planning can be too computationally expensive to perform efficiently. 
 In order to achieve real-time planning, many algorithms use highly simplified model dynamics or kinematics, resulting in a tracking error between the planned path and the true high-dimensional system. 
 This concept is illustrated in Fig. \ref{fig:chasing}, where the path was planned using a simplified planning model, but the real vehicle cannot track this path exactly. 
 In addition, external disturbances (e.g. wind) can be difficult to account for. Crucially, such tracking errors can lead to dangerous situations in which the planned path is safe, but the actual system trajectory enters unsafe regions.
 
 %Real-time planning that is both safe and accurate presents a very difficult challenge: accuracy and robustness in many dyanimcal systme sis difficult to compute, often precluding real-time computer hands.fast planning is generally at odds with the need for maintaining safety and robustness.  

We propose the modular tool FaSTrack: Fast and Safe Tracking, which models the navigation task as a sophisticated \textit{tracking system} that pursues a simplified \textit{planning system}. 
The tracking system accounts for complex system dynamics as well as bounded external disturbances, while the simple planning system enables the use of real-time planning algorithms. 
Offline, a precomputed pursuit-evasion game between the two systems can be analyzed using any suitable method. 
This results in a \textit{tracking error function} that maps the initial relative state between the two systems to the \textit{tracking error bound} (TEB): the maximum possible relative distance that could occur over time. 
This TEB can be thought of as a ``safety bubble" around the planning system that the tracking system is guaranteed to stay within. 
Because the tracking error is bounded in the relative state space, we can precompute and store a \textit{safety control function} that  maps the real-time relative state to the optimal safety control for the tracking system to ``catch" the planning system. 
The offline computations are \textit{independent} of the path planned in real-time.

Online, the autonomous system senses obstacles, which are then augmented by the TEB to ensure that no potentially unsafe paths can be computed. 
Next, a path or trajectory planner uses the simplified planning model to determine the next desired state. 
The tracking system then finds the relative state between itself and the next desired state. 
If this relative state is nearing the TEB then it is plugged into the safety control function to find the instantaneous optimal safety control of the tracking system; otherwise, any controller may be used. In this sense, FaSTrack provides a \emph{least-restrictive} control law. 
This process is repeated as long as desired. 
  
\begin{figure}
	\centering
	\includegraphics[width=0.35\textwidth]{fig/chasing}
	\caption{Left: A planning system (blue disk) using a fast but simple model, followed by a tracking system (green plane) using a more complex model, navigating through an environment with obstacles; the tracking system is guaranteed to stay within some TEB (blue circle). Right: Safety and goal-satisfaction can be guaranteed by planning with respect to augmented obstacles (large blue circles).}
	\label{fig:chasing}
\end{figure}
%

Because we designed FaSTrack to be modular, it can be used with any method for computing the TEB in conjunction with any existing fast path or trajectory planners, enabling motion planning that is real-time, guaranteed safe, and dynamically accurate. 
In this paper, we demonstrate this tool by computing the TEBs solving a Hamilton-Jacobi (HJ) partial differential equation (PDE), and using a different planning algorithm for each numerical example. 
In the three examples, we also consider different models for the tracking system and the planning system.
In the simulations, the system travels through a static environment with obstacles while experiencing disturbances.
 that are only known once they are within the limited sensing range of the vehicle. 
Combining this bound with a kinematic rapidly exploring random trees (RRT) fast path planner \cite{Kuffner2000,Kavraki1996}, the system is able to safely plan and track a trajectory through the environment in real time.