% !TEX root = tracking.tex
\section{Related Work \label{sec:relatedwork}}
Motion planning is a very active area of research in the controls and robotics communities \cite{Hoy2015}.  In this section we will discuss past work on path planning, kinematic planning, and dynamic planning.  A major current challenge is to find an intersection of robust and real-time planning for general nonlinear systems. 
Sample-based planning methods like rapidly-exploring random trees (RRT) \cite{Kuffner2000}, probabilistic road maps (PRM) \cite{Kavraki1996}, fast marching tree (FMT) \cite{Janson2015}, and many others \cite{Richter2016, Karaman2011, Kobilarov2012} can find collision-free paths through known or partially known environments. While extremely effective in a number of use cases, these algorithms are not designed to be robust to model uncertainty or disturbances, and may not even use a dynamic model of the system in the first place.
Motion planning for kinematic systems can also be accomplished through online trajectory optimization using methods such as TrajOpt \cite{Schulman2013} and CHOMP \cite{Ratliff2009}. These methods can work extremely well in many applications, but are generally challenging to implement in real time for nonlinear dynamic systems.

Model predictive control (MPC) has been a very successful method for dynamic trajectory optimization \cite{Qin2003}.  However, combining speed, safety, and complex dynamics is a difficult balance to achieve. Using MPC for robotic and aircraft systems typically requires model simplification to take advantage of linear programming or mixed integer linear programming \cite{Vitus2008, Zeilinger2011, Richter2012}; robustness can also be achieved in linear systems \cite{Richards2006, DiCairano2016}. Nonlinear MPC is most often used on systems that evolve more slowly over time \cite{Diehl2002, Schildbach2016}, with active work to speed up computation \cite{Diehl2009, Neunert2016}. Adding robustness to nonlinear MPC is being explored through algorithms based on minimax formulations and tube MPCs that bound output trajectories around a nominal path (see \cite{Hoy2015} for references).
%speed nonlinear: Findeisen2007,Gupta2015, Torrisi2016

%learning through neural networks \cite{Yan2014}, 
%minimax \cite{Lofberg2003, Kumar2014}
% tube \cite{Mayne2011, Cannon2011, Kumar2014, Gao2014}

There are other methods of dynamic trajectory planning that manage to cleverly skirt the issue of solving for optimal trajectories online.  One such class of methods involve motion primitives \cite{Gillula2010, Dey2016}. Other methods include making use of safety funnels \cite{Majumdar2016}, or generating and choosing random trajectories at waypoints \cite{Kalakrishnan2011, Schwesinger2013}. The latter has been implemented successfully in many scenarios, but is risky in its reliance on finding randomly-generated safe trajectories. 

%%%%%%
One notable real-time planning method that also involves robustness guarantees is given by \cite{KousikVaskovEtAl2017}, in which a forward reachable set for a high-fidelity model of the system is computed offline and then used to prune motion plans generated online using a low-fidelity model. 
The approach relies on an {\em assumed} model mismatch bound; therefore our work has potential to complement works such as \cite{KousikVaskovEtAl2017} by providing the TEB as well as a corresponding feedback tracking controller.
%%%%%

Recent work has considered using offline Hamilton-Jacobi analysis to guarantee tracking error bounds, which can then be used for robust trajectory planning \cite{Bansal2017}. A similar new approach, based on contraction theory and convex optimization, allows computation of offline error bounds that can then define safe tubes around a nominal dynamic trajectory \cite{Singh2017}.

Finally, some online control techniques can be applied to trajectory tracking with constraint satisfaction. For control-affine systems in which a control barrier function can be identified, it is possible to guarantee forward invariance of the desired set through a state-dependent affine constraint on the control, which can be incorporated into an online optimization problem, and solved in real time \cite{Ames2014}. %This can be seen as a method for smoothly blending some of the above safety results with other, performance-based objectives.
%There are also motion planning methods designed to be robust. Control barrier functions \cite{Xu2015, Ames2014} place inequality constraints in the control input that allow for dynamic trajectory planning as a quadratic program. New work in planning using contraction theory works by forming safe tubes online around a nominal dynamic trajectory \cite{Singh2017}. Offline planners like HJ analysis can find control policies and guarantees for nonlinear systems that avoid obstacles and are robust to bounded disturbances \cite{Mitchell05}. %However, this method can only approach real-time speed for very low-dimensional (1D-2D) systems, despite recent work in alleviating the dimensionality hurdle \cite{Chen2016a, Chen2016b}.

The work presented in this paper differs from the robust planning methods above because FaSTrack is designed to be modular and easy to use in conjunction with any path or trajectory planner. Additionally, FaSTrack can handle bounded external disturbances (e.g. wind) and work with both known and unknown environments with obstacles.