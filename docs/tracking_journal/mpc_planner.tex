% !TEX root = tracking.tex
\subsection{8D Quadrotor MPC Example \label{sec:resultsMPC}}

In this section, we demonstrate the online computation framework in Algorithm~\ref{alg:algOnline} with an 8D quadrotor example. Unlike in Sections \ref{sec:reach_planner} and \ref{sec:resultsRRT}, we consider a time-varying TEB and utilize the Model Predictive Control (MPC) technique as the online planner. 

First we define the 8D dynamics of the tracking quadrotor, and the 4D dynamics of a double integrator, which serves as the planning system to be used in MPC:

\begin{equation}
\label{eq:Quad8D_dyn}
\begin{aligned}
\begin{array}{c}
\left[
\begin{array}{c}
\dot x\\
\dot v_x\\
\dot \theta_x\\
\dot \omega_x\\
\dot y\\
\dot v_y\\
\dot \theta_y\\
\dot \omega_y
\end{array}
\right]
=
\left[
\begin{array}{c}
v_{x,s} + d_x\\
g \tan \theta_x\\
-d_1 \theta_x + \omega_x\\
-d_0 \theta_x + n_0 a_x\\
v_y\\
g \tan \theta_y + \MCnote{d_{v_y}}\\
-d_1 \theta_y + \omega_y\\
-d_0 \theta_y + n_0 a_y
\end{array}
\right], \quad
\left[
\begin{array}{c}
\dot {\hat x}\\
\dot {\hat v}_x\\
\dot {\hat y}\\
\dot {\hat v}_y\\
\end{array}
\right] 
=
\left[
\begin{array}{c}
\hat v_x\\
\hat a_x\\
\hat v_y\\
\hat a_y\\
\end{array}
\right],
\end{array}\\
\end{aligned}
\end{equation}
where the states and controls of the 8D system are defined the same as \eqref{eq:Quad10D_dyn}. The position $(\hat x,\hat y)$ and velocity $(\hat v_x, \hat v_y)$ are the states of the 4D system. The controls are $(\hat a_x, \hat a_y)$, which represent the acceleration in each positional dimension. Note that the disturbances in the 8D system $(d_{v_x},d_{v_y})$ act on the \MCnote{acceleration} in each direction. \MCnote{***Explain where the disturbances come from (physically)***}

\subsubsection{Offline precomputation}

The relative dynamics between the 8D and the 4D system are defined using (\ref{eq:rdyn}):
%
% \begin{equation}
% \begin{aligned}
% \dot x_r &= \dot x - \dot{\hat x} = v_x - \hat v_x = v_{x,r}\\
% \dot v_{x,r} &= \dot v_x - \dot{\hat v}_x = g \tan \theta_x - \hat a_x  + \MCnote{d_{v_x}}\\
% \dot y_r &= \dot y - \dot{\hat y} = v_y - \hat v_y = v_{y,r}\\
% \dot v_{y,r} &= \dot v_y - \dot{\hat v}_y = g \tan \theta_y - \hat a_y  + \MCnote{d_{v_y}}\\
% \end{aligned}
% \end{equation}
%
\begin{equation}
\label{eq:Quad8DRel_dyn}
\begin{aligned}
\begin{array}{c}
\left[
\begin{array}{c}
\dot x_r\\
\dot v_{x,r}\\
\dot \theta_x\\
\dot \omega_x\\
\dot y_r\\
\dot v_{y,r}\\
\dot \theta_y\\
\dot \omega_y\\
\end{array}
\right]
=
\left[
\begin{array}{c}
v_{x,r}\\
g \tan \theta_x - \hat a_x + d_{v_x}\\
-d_1 \theta_x + \omega_x\\
-d_0 \theta_x + n_0 a_x\\
v_{y,r}\\
g \tan \theta_y - \hat a_y + d_{v_y}\\
-d_1 \theta_y + \omega_y\\
-d_0 \theta_y + n_0 a_y\\
\end{array}
\right],
\end{array}\\
\end{aligned}
\end{equation}
where the states $(x_r,v_{x,r},y_r,v_{y,r})$ are the position and velocity of the 8D quadrotor in the frame of the double integrator. The remaining states $(\theta_x,\omega_x,\theta_y,\omega_y)$ are the pitch (rate) and roll (rate) of the 8D quadrotor.

\MCnote{Added offline reachability computation results and the range of velocity and positional TEBs.}   
\color{blue} notations for time-varying TEBs and Minkowski addition?modify Alg. 1\color{black}

\subsubsection{Online computation}
%
We utilize the MPC design introduced in \cite{Zhang2017} for the online path planning. See in Problem~\ref{pr: MPC}.
%
\begin{problem}\label{pr: MPC}
\begin{align*}
\min_{\mathbf{p},\mathbf{u}}  & \quad \sum^{N-1}_{k=0} l(p_k,u_k) + l_f(p_N-p_f)  \\
s.t. \quad & p_0 = p_{init},\\
&p_{k+1} = f_p(p_k,u_k),\\
& p_k \in \mathbb{P}_k ,\enspace u_k \in \mathbb{U},\\
& \mathbb{S}(p_k)\cap\constrAug(t_k) = \emptyset
\end{align*}
\end{problem}
where $l(\cdot,\cdot)$ and $l_f(\cdot)$ are convex stage and terminal cost functions, $N$ is the horizon for the MPC problem, and $t_k = t_0 + k \MCnote{\Delta t}$ denotes the current time used in simulation, with $t_0$ being the time when the MPC problem starts to be solved and $\Delta t$ the MPC sampling interval. The dynamical system $f_p(\cdot,\cdot)$ is set to be a discretized model of the 4D dynamics in \eqref{eq:Quad8D_dyn}. The state and input sequences along the horizon  are denoted by $\mathbf{p}=[p^{T}_0,p^{T}_1,\cdots,p^{T}_N]^{T}$ and $\mathbf{u}=[u^{T}_0,u^{T}_1,\cdots,u^{T}_{N-1}]^{T}$. The velocity states are subject to convex time-varying constraints:
%
\begin{equation}
(\hat v_{x,k},\hat v_{y,k}) \subset p_k \in \mathbb{P}_k :=\mathbb{P}\oplus\TEB_\pstate(t_k) \enspace ,
\end{equation}
%
where $\oplus$ denotes the Minkowski addition, $\mathbb{P}$ denotes the original state constraint, and $\TEB_\pstate(t_k)$ is the tracking error bound sampled at $t_k$, respectively. Given the state vector $p_k$, we denote the position of the controlled object by $(\hat x_k,\hat y_k) := \mathbb{S}(p_k)\subseteq \mathbb{R}^{2}$. To avoid collision with obstacles, $\mathbb{S}(p_k)$ is subject to the following constraint: 
%
\begin{equation}
\mathbb{S}(p_k)\cap\constrAug(t_k) = \emptyset \enspace ,
\end{equation}
%
with 
%
\begin{equation}
\constrAug(t_k) := \constrSense(t_0)\oplus\TEB_\pstate(t_k) \enspace .
\end{equation}
%
where $\constrSense(t_0)$ denotes the obstacles sensed at $t_0$.

In this paper, we represent the obstacles as polytopes, i.e., $\constrSense = \cap \constr^{i}$ with $\constr^{i}:= \{z\in\mathbb{R}^{n} \mid A^{i}z\leq b^{i}\}$ for $i = 1,\cdots ,M$. We follow the approach presented in \cite{Zhang2017} to compute a local minimal solution, by involving extra variables $\lambda^{i}$ for each obstacle $\constr^{i}$ and reformulating the collision avoidance constraint equivalently as follows: 
%
\begin{equation}
\exists \lambda^{i} >0, \; \mbox{s.t.} \; (A^{i} \mathbb{S}(p_k) - b^{i})^{T}\lambda^{i}  > 0, \; \|A^{i^{T}}\lambda^{i}\|_2\leq 1\enspace .
\end{equation}
%
Note that the collision avoidance constraint causes the MPC problem to be non-convex and thus computationally expensive.

The procedure of finding the next state of the planning system using the proposed MPC planner is summarized in Algorithm \ref{alg:mpc}.
%
\begin{algorithm}	
	\caption{MPC Path Planner Block}
	\label{alg:mpc}
	\begin{algorithmic}[1]
		\STATE \textbf{Initialization}:
 		\STATE Set initial time and states: $t_0 \leftarrow \tvar, \pstate_0 \leftarrow \pstate$
		\IF{MPC is ready to re-plan}
			\STATE Solve MPC for the optimal control sequence: $\mathbf{u}_t \leftarrow \text{Problem\ref{pr: MPC}} (t_0,p_0,\constrAug)$
		\ENDIF
        \STATE Get the current control: $u(t) \leftarrow u_k \in \mathbf{u}_t$ such that $t \in [t_0 + k \Delta t, t_0 + (k+1) \Delta t]$
		\STATE Output the next state: $\pstate_\text{next} = f_p(p,u(t))$
	\end{algorithmic}
\end{algorithm}

\subsubsection{Simulations}

The values for parameters $d_0,d_1,n_0,k_T,g$ used for the 8D model are: $d_0=10,d_1=8,n_0=10,k_T=0.91,g=9.81$.

The 8D control bounds are $|a_x|,|a_y|\leq10$ degrees.

The 4D control bound is $\|(\hat a_x,\hat a_y)\|_2\leq1.0$ m/s$^{2}$.

The disturbance bounds are $|d_{v_x}|,|d_{v_y}|\leq0.1$ m/s.

The sensing range is $\MCnote{r = 5}$ meters.

For the MPC problem we used a horizon $N=8$ with sampling interval $\Delta t = 0.2$ s.

Implementation of the MPC planner was based on MATLAB and \texttt{ACADO Toolkit} \cite{Houska2011a}. The nonlinear MPC problem was solved using an online active set strategy implemented in \texttt{qpOASES} \cite{Ferreau2014}. All the simulation results were obtained on a laptop with Ubuntu 14.04 LTS operating system and a Core i5-4210U CPU. The MPC planner re-plans every 0.8 s with an average computational time of 0.37 s for each planning loop. The frequency of control was once every 0.1 s for the 8D quadrotor system.

\MCnote{Simulation figures and explanations to be added.}