% !TEX root = tracking.tex
\section{8D Quadrotor MPC Example \label{sec:results}}
%
In this section, we demonstrate the online computation framework in Algorithm~\ref{alg:algOnline} with a 8D quadrotor example. Different from the experiment in Section~\ref{sec:results}, we consider a time-varying tracking error bound and utilize the Model Predictive Control (MPC) technique as the online planner. 
%
\subsection{Add time-varying obstacle size}
\color{blue} notations for time-varying tracking error bounds and Minkowski addition?modify Alg. 1\color{black}

The time-varying tracking error bound $\{\mathcal{B}_{p,t}$ consists of three elements:
\begin{equation}
\mathcal{B}_{p,t} :=\{\mathcal{X}_{aug,t}, \mathcal{U}_{aug,t}, \mathcal{S}_{aug,t}\}\enspace ,
\end{equation}
where the subscript $t$ denotes for the time index. The elements $\mathcal{X}_{aug,t}$,  $\mathcal{U}_{aug,t}$ and  $\mathcal{S}_{aug,t}$ represent the error bounds for state, input and obstacles at time $t$, respectively.
%
\subsection{Online planner based on MPC}
%
We utilize the MPC design introduced in \cite{zhang_2017_MPC}, given in Problem~\ref{pr: MPC}, for the online path planning.
%
\begin{problem}\label{pr: MPC}
\begin{align*}
\min_{x,u}  & \quad \sum^{N}_{t=0} l(x_t,u_t)  \\
s.t. \quad &x_{t+1} = f_p(x_t,u_t)\\
& x_t \in \mathbb{X}_t\quad u_t \in \mathbb{U}_t\\
& \mathbb{S}(x_t)\cup\mathcal{O}_{aug,t} = \emptyset,\\
& x_N = x_f ,\quad x(0) = \bar{x} \enspace.
\end{align*}
\end{problem}
\noindent where $l_i(\cdot,\cdot)$ is a  convex stage cost function and $N$ is the horizon for the MPC problem. The dynamical system $f_p(\cdot,\cdot)$ is set to be a discretized model of the 4D dynamics in \ref{eq:Quad8D}. The state and input sequences along the horizon  are denoted by $x=[x^{T}_0,x^{T}_1,\cdots,x^{T}_N]^{T}$ and $u=[u^{T}_0,u^{T}_1,\cdots,u^{T}_{N-1}]^{T}$. The states and inputs are subject to convex time-varying constraints:
%
\begin{equation}
x_t \in \mathbb{X}_t :=\mathbb{X}\oplus\mathcal{X}_{aug,t} \quad u_t \in\mathbb{U}_t := \mathbb{U}\oplus\mathcal{U}_{aug,t} \enspace ,
\end{equation}
%
where $\oplus$ denotes the Minkowski addition, and $\mathbb{X}$ and $\mathbb{U}$ denote the original state and input constraints, respectively. Given the state $x_t$, we denote the position of the controlled objective by $\mathbb{S}(x_t)\subset \mathbb{R}^{2}$. To avoid obstacle collision, the state $x_t$ is also subject to the following constraint: 
%
\begin{equation}
\mathbb{S}(x_t)\cup\mathcal{O}_{aug,t} = \emptyset \enspace ,
\end{equation}
%
with 
%
\begin{equation}
\mathcal{O}_{aug,t} := \obsSense\oplus\mathcal{S}_{aug,t} \enspace .
\end{equation}
%
where the symbol $\oplus$ denotes for the Minkowski addition.
%
In this paper, we represent the obstacles as polytopes, i.e., $\obsSense = \cap \mathcal{O}^{i}$ with $\mathcal{O}^{i}:= \{x\in\mathbb{R}^{n} \mid A^{i}x\leq b^{i}\}$ for $i = 1,\cdots ,M.$. Therefore, the collision avoidance constraint is non-convex and computationally difficult. We follow the approach presented in \cite{zhang_2017_MPC} to compute a local minimal solution, by involving an extra variable $\lambda^{i}$ for each obstacle and reformulating the collision avoidance constraint equivalently as follows: 
%
\begin{equation}
\exists \lambda^{i} >0, \; \mbox{s.t.} \; (A^{i}x- b^{i})^{T}\lambda^{i}  > 0, \; \|A^{i^{T}}\lambda^{i}\|\leq 1\enspace .
\end{equation}
%
\subsubsection{Implement of the MPC planner with ACADO Toolbox}
 

\subsection{Precomputation of 10D-4D system}
First we define the 10D dynamics of the tracking quadrotor and the simple 4D dynamics of a quadrotor:

\MCnote{Note that disturbance is now applied to the acceleration instead of velocity}

\begin{equation}
\label{eq:Quad8D}
\begin{aligned}
\begin{array}{c}
\left[
\begin{array}{c}
\dot x_s\\
\dot v_{x,s}\\
\dot \theta_x\\
\dot\omega_x\\
\dot y_s\\
\dot v_{y,s}\\
\dot\theta_y\\
\dot\omega_y
\end{array}
\right]
=
\left[
\begin{array}{c}
v_{x,s} + d_x\\
g \tan \theta_x\\
-d_1 \theta_x + \omega_x\\
-d_0 \theta_x + n_0 a_x\\
v_{y,s} + d_y\\
g \tan \theta_y \\
-d_1 \theta_y + \omega_y\\
-d_0 \theta_y + n_0 a_y
\end{array}
\right],
\left[
\begin{array}{c}
\dot{x_p}\\
\dot v_{x,p}\\
\dot{y,p}\\
\dot v_{y,p}\\
\end{array}
\right] 
=
\left[
\begin{array}{c}
v_{x,p}\\
a_x\\
v_{y,p}\\
a_y\\
\end{array}
\right]
\end{array}\\
\end{aligned}
\end{equation}
where states $(x, y, z)$ denote the position, $(v_x, v_y, v_z)$ denote the velocity, $(\theta_x, \theta_y)$ denote the pitch and roll, and $(\omega_x, \omega_y)$ denote the pitch and roll rates. The controls of the 10D system are $(a_x, a_y, a_z)$, where $a_x$ and $a_y$ represent the desired pitch and roll angle, and $a_z$ represents the vertical thrust. The 3D system controls are $(b_x, b_y, b_z)$, and represent the velocity in each positional dimension. The disturbances in the 10D system $(d_x, d_y, d_z)$ are caused by wind, which acts on the velocity in each dimension. Note that the states of the 3D dynamics are a subset of the 10D state space; the matrix Q used in the online computation matches the position states of both systems. Next the relative dynamics between the two systems is defined using (\ref{eq:rdyn}):

\begin{equation}
\begin{aligned}
\dot x_r &= \dot x_s - \dot x_p = v_{x,s} - v_{x,p} = v_{x,r} + d_x\\
\dot v_{x,r} &= \dot v_{x,s} - \dot v_{x,p} = g \tan \theta_x - a_x\\
\dot y_r &= \dot y_s - \dot y_p = v_{y,s} - v_{y,p} = v_{y,r} + d_y\\
\dot v_{y,r} &= \dot v_{y,s} - \dot v_{y,p} = g \tan \theta_y - a_y\\
\end{aligned}
\end{equation}

\begin{equation}
\label{eq:Quad8DRel_dyn}
\begin{aligned}
\begin{array}{c}
\left[
\begin{array}{c}
\dot{x_r}\\
\dot v_{x,r}\\
\dot{\theta_{x}}\\
\dot\omega_{x}\\
\dot{y_r}\\
\dot{v_{y}}\\
\dot{\theta_{y}}\\
\dot\omega_{y}\\
\end{array}
\right]
=
\left[
\begin{array}{c}
v_{x,r} + d_x\\
g \tan \theta_x - a_x\\
-d_1 \theta_x + \omega_x\\
-d_0 \theta_x + n_0 u_x\\
v_{y,r} + d_y\\
g \tan \theta_y - a_y\\
-d_1 \theta_y + \omega_y\\
-d_0 \theta_y + n_0 u_y\\
\end{array}
\right]
\end{array}\\
\end{aligned}
\end{equation}
The values for parameters $d_0,d_1,n_0,k_T,g$ used were: $d_0=10,d_1=8,n_0=10,k_T=0.91,g=9.81$. The 10D control bounds were $|a_x|,|a_y|\leq10$ degrees, $0\leq a_z\leq 1.5g$ m/s$^{2}$. The 3D control bounds were $|b_x|,|b_y|,|b_z|\leq0.5$ m/s. The disturbance bounds were $|d_x|,|d_y|,|d_z|\leq0.1$ m/s.

\subsubsection{Online planner based on MPC}
%
We utilize the MPC design introduced in \cite{zhang_2017_MPC} for the online path planning. See in Problem~\ref{pr: MPC}.
%
\begin{problem}\label{pr: MPC}
\begin{align*}
\min_{x,u}  & \quad \sum^{N-1}_{t=0} l(x_t,u_t)  \\
s.t. \quad &x_{t+1} = f_p(x_t,u_t)\\
& x_t \in \mathbb{X}_t\quad u_t \in \mathbb{U}_t\\
& \mathbb{S}(x_t)\cup\mathcal{O}_{aug,t} = \emptyset,\\
& x_N = x_f ,\quad x(0) = \bar{x} \enspace.
\end{align*}
\end{problem}
\noindent where $l_i(\cdot,\cdot)$ is a  convex stage cost function and $N$ is the horizon for the MPC problem. The dynamical system $f_p(\cdot,\cdot)$ is set to be a discretized model of the 4D dynamics in \ref{eq:Quad8D}. The state and input sequences along the horizon  are denoted by $x=[x^{T}_0,x^{T}_1,\cdots,x^{T}_N]^{T}$ and $u=[u^{T}_0,u^{T}_1,\cdots,u^{T}_{N-1}]^{T}$. The states and inputs are subject to convex time-varying constraints:
%
\begin{equation}
x_t \in \mathbb{X}_t :=\mathbb{X}\oplus\TEB_\pstate(t) \quad u_t \in\mathbb{U}_t := \mathbb{U}\oplus\TEB_\pstate(t)  \enspace ,
\end{equation}
%
where $\oplus$ denotes the Minkowski addition, and $\mathbb{X}$ and $\mathbb{U}$ denote the original state and input constraints, respectively. Given the state $x_t$, we denote the position of the controlled objective by $\mathbb{S}(x_t)\subset \mathbb{R}^{2}$. To avoid obstacle collision, the state $x_t$ is also subject to the following constraint: 
%
\begin{equation}
\mathbb{S}(x_t)\cup\mathcal{O}_{aug,t} = \emptyset \enspace ,
\end{equation}
%
with 
%
\begin{equation}
\mathcal{O}_{aug,t} := \obsSense\oplus\TEB_\pstate(t) \enspace .
\end{equation}
%
In this paper, we represent the obstacles as polytopes, i.e., $\obsSense = \cap \mathcal{O}^{i}$ with $\mathcal{O}^{i}:= \{x\in\mathbb{R}^{n} \mid A^{i}x\leq b^{i}\}$ for $i = 1,\cdots ,M.$. Therefore, the collision avoidance constraint is non-convex and computationally expensive. We follow the approach presented in \cite{zhang_2017_MPC} to compute a local minimal solution, by involving extra variable $\lambda^{i}$ for each obstacle and reformulating the collision avoidance constraint equivalently as follows: 
%
\begin{equation}
\exists \lambda^{i} >0, \; \mbox{s.t.} \; (A^{i}x- b^{i})^{T}\lambda^{i}  > 0, \; \|A^{i^{T}}\lambda^{i}\|\leq 1\enspace .
\end{equation}
%
\subsubsection{Implement of the MPC planner with ACADO Toolbox}
 