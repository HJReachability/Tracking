% !TEX root = tracking.tex
\subsection{8D quadrotor-4D double integrator example with MPC \label{sec:resultsMPC}}

In this section, we demonstrate the online computation framework in Algorithm \ref{alg:algOnline} with an 8D quadrotor example. 
Unlike in Sections \ref{sec:reach_planner} and \ref{sec:resultsRRT}, we consider a time-varying TEB and utilize MPC as the online planner.
In addition, the TEB depends on both position and speed, as opposed to just position.

First we define the 8D dynamics of the near-hover quadrotor, and the 4D dynamics of a double integrator, which serves as the planning system to be used in MPC:

\begin{equation}
\label{eq:Quad8D_dyn}
\begin{bmatrix}
\dot x\\
\dot v_x\\
\dot \theta_x\\
\dot \omega_x\\
\dot y\\
\dot v_y\\
\dot \theta_y\\
\dot \omega_y
\end{bmatrix} =
\begin{bmatrix}
v_{x,s} + d_x\\
g \tan \theta_x\\
-d_1 \theta_x + \omega_x\\
-d_0 \theta_x + n_0 a_x\\
v_y + d_y\\
g \tan \theta_y\\
-d_1 \theta_y + \omega_y\\
-d_0 \theta_y + n_0 a_y
\end{bmatrix}, \quad
\begin{bmatrix}
\dot {\hat x}\\
\dot {\hat v}_x\\
\dot {\hat y}\\
\dot {\hat v}_y\\
\end{bmatrix} =
\begin{bmatrix}
\hat v_x\\
\hat a_x\\
\hat v_y\\
\hat a_y\\
\end{bmatrix},
\end{equation}

\noindent where the states, controls, and disturbances are the same as the first 8 components of the dynamics in \eqref{eq:Quad10D_dyn}. 
The position $(\hat x,\hat y)$ and velocity $(\hat v_x, \hat v_y)$ are the states of the 4D system. 
The controls are $(\hat a_x, \hat a_y)$, which represent the acceleration in each positional dimension. 

The model parameters are chosen to be $d_0=10$, $d_1=8$, $n_0=10$, $k_T=0.91$, $g=9.81$, $|u_x|, |u_y| \le \pi/9$, $|\hat a_x|, |\hat a_y| \le 1$, $|\dstb_x|, |\dstb_y| \le 0.2$.

\subsubsection{Offline precomputation}
We define the relative system states to be the error states $(x_r, v_{x,r}, y_r, v_{y,r})$, which are the relative position and velocity, concatenated with the rest of the states in the 8D system.
Defining $\rtrans = \mathbf I_8$ and 

\begin{equation*}
\ptmat = 
\begin{bmatrix}
  \begin{bmatrix} 1 \\ \mathbf 0_{3 \times 1} \end{bmatrix} 
    & \mathbf 0_{4\times 1} \\
  \mathbf 0_{4\times 1} 
    & \begin{bmatrix} 1 \\ \mathbf 0_{3 \times 1} \end{bmatrix} 
\end{bmatrix},
\end{equation*}

\noindent we obtain the following relative system dynamics:
\begin{equation}
\label{eq:Quad8DRel_dyn}
\begin{bmatrix}
\dot x_r\\
\dot v_{x,r}\\
\dot \theta_x\\
\dot \omega_x\\
\dot y_r\\
\dot v_{y,r}\\
\dot \theta_y\\
\dot \omega_y\\
\end{bmatrix} =
\begin{bmatrix}
v_{x,r} + \dstb_x\\
g \tan \theta_x - \hat a_x\\
-d_1 \theta_x + \omega_x\\
-d_0 \theta_x + n_0 a_x\\
v_{y,r} + \dstb_y\\
g \tan \theta_y - \hat a_y\\
-d_1 \theta_y + \omega_y\\
-d_0 \theta_y + n_0 a_y\\
\end{bmatrix}.
\end{equation}

As in the 10D-3D example in Section \ref{sec:resultsRRT}, the relative dynamics are decomposable into two 4D subsystems, and so computations were done in 4D space.

Fig. \ref{fig:vf_TEB:8D4D} shows the $(x_r, v_{x,r})$-projection of value function across several different times on the left subplot.
The total time horizon was $T=15$, and the value function did not converge.
The gray horizontal plane indicates the value of $\underline V$, which was $1.14$.
Note that with increasing $\tau$, $V(\rstate,\thor-\tau)$ is non-increasing, as proven in Proposition \ref{prop:nonconv}.

The right subplot of Fig. \ref{fig:vf_TEB:8D4D} shows the $(x_r, v_{x,r})$-projection of the time-varying TEB.
At $\tau=0$, the TEB is the smallest, and as $\tau$ increases, the size of TEB also increases, consistent with Proposition \ref{prop:nonconv}.
In other words, the set of possible error states $(x_r, v_{x,r})$ in the relative system increases with time, which makes intuitive sense.

The TEB shown in Fig. \ref{fig:vf_TEB:8D4D} are used to augment planning constraints in the $\hat x$ and $\hat v_x$ dimensions.
Since we have chosen identical parameters for the first four and last four states, the TEB in the $\hat y$ and $\hat v_y$ dimensions is identical.

On a desktop computer with an Intel Core i7 5820K CPU, the offline computation on a $81\times81\times65\times65$ grid with $75$ time discretization points took approximately 30 hours and  required approximately 17 GB of RAM using a C++ implementation of level set methods for solving \eqref{eq:HJVI}.
Note that unlike the other numerical examples, look-up tables representing the value function and its gradient must be stored at each time discretization point.

\subsubsection{Online sensing and planning}
%
We utilize the MPC design introduced in \cite{Zhang2017} for the online trajectory planning. The MPC formulation is given in Problem~\ref{pr: MPC}.
%
\begin{problem}\label{pr: MPC}
\begin{align*}
\min_{\mathbf{p},\mathbf{u}}  & \quad \sum^{N-1}_{k=0} l(p_k,u_k) + l_f(p_N-p_f)  \\
s.t. \quad & p_0 = p_{init},\\
&p_{k+1} = h(p_k,u_k),\\
& u_k \in \mathbb{U},\enspace p_k \in \constrAug(t_k) 
\end{align*}
\end{problem}
%
where $l(\cdot,\cdot)$ and $l_f(\cdot)$ are convex stage and terminal cost functions and $N$ is the horizon for the MPC problem. $t_k = t_0 + k \MCnote{\Delta t}$ denotes for the time index along the MPC horizon with $t_0$ and $\Delta t$ being the initial time step and the sampling interval, respectively. Note that the horizon $N$ and the sampling interval $\Delta t$ are selected such that $t_0 + N \MCnote{\Delta t}\leq \tau - T$ with $\tau$ given by Step~9 in Algorithm~\ref{alg:algOnline} and $T$ defined for TEB in (\ref{eq:TEB}). The initial state is denote by $p_{init}$. The dynamical system $h(\cdot,\cdot)$ is set to be a discretized model of the 4D dynamics in \eqref{eq:Quad8D_dyn}. The state and input constraints are $\mathbb{U}$ and $\constrAug(t_k)$, respectively. Note that the time-varying constraint $\constrAug(t_k)$ contains the augmented state constraints:
%
\begin{equation}
p_k \in \mathbb{P}_k :=\mathbb{P}\ominus\TEB_\estate(t_k) \enspace ,
\end{equation}
%
where $\mathbb{P}$ denotes the original state constraint, and $\TEB_\estate(t_k)$ is the tracking error bound at $t_k$, and the additional constraints for collision avoidance:
%
\begin{equation}
\mathbb{S}(p_k)\cap\mathbb{O}\oplus\mathbb{S}(\TEB_\estate(t_k)) = \emptyset \enspace ,
\end{equation}
%
where the operator $\mathbb{S}(\cdot)$ abstracts the position of the controlled objective from the state $p_k$, i.e., $(\hat x_k,\hat y_k) := \mathbb{S}(p_k)\subseteq \mathbb{R}^{2}$, and $\mathbb{O}$ denotes the union set of the sensed obstacles.
%
\begin{remark}
In this paper, we represent the obstacles as polytopes, i.e., $\mathbb{O} = \cup \mathbb{O}^{i}$ with $\mathbb{O}^{i}:= \{z\in\mathbb{R}^{n} \mid A^{i}z\leq b^{i}\}$ for $i = 1,\cdots ,M$. Due to the collision avoidance constraint, the MPC problem becomes non-convex and thus computationally expensive. We follow the approach presented in \cite{Zhang2017} to compute a local minimal solution, by involving extra variables $\lambda^{i}$ for each obstacle $\mathbb{O}^{i}$ and reformulating the collision avoidance constraint equivalently as follows: 
%
\begin{equation}
\exists \lambda^{i} >0, \; \mbox{s.t.} \; (A^{i} \mathbb{S}(p_k) - b^{i})^{T}\lambda^{i}  > 0, \; \|A^{i^{T}}\lambda^{i}\|_2\leq 1\enspace .
\end{equation}
%
\end{remark}
%
The procedure of finding the next tracking state using the MPC-based planner is summarized in Algorithm~\ref{alg:mpc}.
%
\begin{algorithm}	
	\caption{MPC Path Planner}
	\label{alg:mpc}
	\begin{algorithmic}[1]
 		\STATE Initialize the time and state: $t_0 \leftarrow \tvar, \pstate_0 \leftarrow \pstate$
		\IF{MPC is ready to re-plan}
			\STATE Solve Problem~\ref{pr: MPC} with the inputs $t_0$, $p_{init}$ and $\constrAug$
		\ENDIF
          \STATE Output the next planed state: $p_{t_1}$
	\end{algorithmic}
\end{algorithm}

\subsubsection{Simulations}

%The values for parameters $d_0,d_1,n_0,k_T,g$ used for the 8D model are: $d_0=10,d_1=8,n_0=10,k_T=0.91,g=9.81$.

%The 8D control bounds are $|a_x|,|a_y|\leq10$ degrees.

%The 4D control bound is $\|(\hat a_x,\hat a_y)\|_2\leq1.0$ m/s$^{2}$.

%The disturbance bounds are $|d_{v_x}|,|d_{v_y}|\leq0.1$ m/s.

%The sensing range is $\MCnote{r = 5}$ meters.

For the MPC problem we used a horizon $N=8$ with sampling interval $\Delta t = 0.2$ s.

Implementation of the MPC planner was based on MATLAB and \texttt{ACADO Toolkit} \cite{Houska2011a}. The nonlinear MPC problem was solved using an online active set strategy implemented in \texttt{qpOASES} \cite{Ferreau2014}. All the simulation results were obtained on a laptop with Ubuntu 14.04 LTS operating system and a Core i5-4210U CPU. The MPC planner re-plans every 0.8 s with an average computational time of 0.37 s for each planning loop. The frequency of control was once every 0.1 s for the 8D quadrotor system.

\MCnote{Simulation figures and explanations to be added.}