% !TEX root = tracking.tex
\section{Numerical examples} \label{sec:results}

In this section, we demonstrate the FaSTrack framework in three numerical examples involving a 5D car tracking a Dubins car model with the FSM planner, a 10D quadrotor tracking a single integrator model with the RRT planner, and an 8D quadrotor tracking a double integrator model with the MPC planner.
In each example, obstacles in the environment are \textit{a priori} unknown, and are revealed to the vehicle when they are sensed.
Whenever, obstacle map is updated, the planner re-plans a trajectory in real-time.
In this paper, the details of sensing are kept as simple as possible; we aim to only demonstrate our framework for real-time guaranteed safe planning and re-planning.
In general, any other planner can be used for planning in unknown environments, as long as planning and re-planning can be done in real-time.

For each example, we first describe the tracking and planning models. 
Next, we present the relative dynamics as well as the precomputation results. 
Afterwards, we briefly describe the planning algorithm and how obstacles are sensed by the vehicle. 
Finally, we show trajectory simulation results.

\subsection{5D-3D example with FSM planner \label{sec:reach_planner}}

For our first example, we demonstrate the combination of fast planning and provably robust tracking by combining the fast sweeping method (FSM) \cite{} with our computed TEB. 
FSM is an efficient optimal control-based planner for car-like systems, and provides the globally optimal trajectory in real-time.

Consider the 5D car model and the Dubins car dynamics as follows:

\begin{equation}
\label{eq:5D_and_3D_dyn}
\begin{aligned}
\begin{array}{c}
\left[
\begin{array}{c}
\dot x\\
\dot y\\
\dot\theta\\
\dot v\\
\dot \omega
\end{array}
\right]
=
\left[
\begin{array}{c}
v \cos \theta + d_x\\
v \sin \theta + d_y\\
\omega \\
a\\
\alpha
\end{array}
\right], \quad
\left[
\begin{array}{c}
\dot {\hat x}\\
\dot {\hat y}\\
\dot {\hat \theta}\\
\end{array}
\right] 
=
\left[
\begin{array}{c}
\hat v \cos \hat\theta\\
\hat v \sin \hat\theta\\
\hat \omega
\end{array}
\right],
\end{array}\\
\end{aligned}
\end{equation}

\noindent where $(x,y,\theta),(\hat x, \hat y, \hat\theta)$ represent the pose (position and heading) of the 5D car model and the Dubins car model respectively. The speed and turn rate $(v, \omega)$ are states for the 5D car model; for the Dubins car the speed $\hat v$ is a constant, and the turn rate $\hat \omega$ is the control. The control of the 5D car consists of the linear and angular acceleration, $(a, \alpha)$.

In this example, we use FSM \cite{} to perform real-time planning for the Dubins car model, whose trajectory is tracked by the 5D car model.

\subsubsection{Offline computation}

We define a coordinate system $(x_r, y_r, \theta_r, v, \omega)$ such that $(x_r, y_r, \theta_r)$ is the position and heading of the 5D car in the frame of the Dubins car, as shown in Figure \cite{}, and $(v, \omega)$ represents the speed and turn rate of the 5D car. Following \cite{Mitchell05} for the time derivative of $(x_r, y_r, \theta_r)$, we obtain the following relative dynamics:

\begin{equation}
\label{eq:5D_and_3D_rdyn}
\begin{aligned}
\left[
\begin{array}{c}
\dot x_r\\
\dot y_r\\
\dot\theta_r\\
\dot v\\
\dot \omega
\end{array}
\right]
=
\left[
\begin{array}{c}
- \hat v + v \cos \theta_r + \hat \omega y_r + d_x\\
v \sin \theta_r - \hat \omega x_r + d_y\\
\omega - \hat \omega \\
a\\
\alpha
\end{array}
\right].
\end{aligned}
\end{equation}

\MCnote{Show tracking error bound}

\MCnote{Offline computation details (time etc.)}

\subsubsection{Online sensing and planning}
Besides planning a trajectory to the goal, the car must also sense obstacles in the vicinity. For illustration, we chose a simple virtual sensor that reveals obstacles within a range of $r$ meters.

Once an obstacle is sensed, the FSM planner replans the trajectory while taking into account all obstacles that have been sensed so far. 
To ensure that the 5D car does not collide with the obstacles despite error in tracking, planning is done with respect to augmented obstacles that are ``expanded'' from the sensed obstacles by $\underline\valfunc$ in $(x,y)$ position space.

\MCnote{Online computation details (time etc.)}

\subsubsection{Simulations}

The Dubins car parameters are

The control bounds for the 5D car are $a \in []$, $|\alpha| < $.

The disturbance bound is $\|(d_x, d_y)\|_2 < $.

The sensing range is $r = $ meters.

We used a C implementation of FSM.

Computation times:
\begin{itemize}
\item === TOTAL ===
\item  512 iterations in 637.5527 seconds
\item 1.2452 seconds per iteration on average
\item === PLANNING ===
\item 512 iterations in 33.5683 seconds
\item 0.0656 seconds per iteration on average
\item === TRACKING ===
\item 512 iterations in 0.8991 seconds
\item 0.0018 seconds per iteration on average
\end{itemize}

On an unoptimized MATLAB implementation on a desktop computer with a Core i7-2600K CPU, each iteration took approximately ?? ms on average. Most of this time is spent on planning: obtaining the tracking controller took approximately ?? ms per iteration on average. The frequency of control was once every ?? ms.

\MCnote{Fig. ?? shows the simulation results. 
Four time snapshots are shown. The initial position is $(-12, 0, 0)$, and the goal position is $(12, 0, 0)$. The top left subplot shows the entire trajectory from beginning to end. In all plots, a magenta star represents the position of the planning model; its movement is based on the paths planned by RRT, and is modeled by a 3D holonomic vehicle with a maximum speed. The blue box around the magenta star represents the tracking error bound.}