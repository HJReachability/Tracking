%%%%%%%%%%%%%%%%%%%%%%%%%%%%%%%%%%%%%%%%%%%%%%%%%%%%%%%%%%%%%%%%%%%%%%%%%%%%%%%%
%2345678901234567890123456789012345678901234567890123456789012345678901234567890
%        1         2         3         4         5         6         7         8

%\documentclass[12pt, draftcls, onecolumn]{IEEEtran}
\documentclass[journal]{IEEEtran}                                                   

 
\IEEEoverridecommandlockouts                              % This command is only
                                                          % needed if you want to
                                                          % use the \thanks command

\usepackage{amsmath}    % need for sub equations
\usepackage{amsfonts}
\usepackage{graphicx}   % need for figures
\usepackage{subcaption}
\usepackage{epsfig} 
\usepackage{algorithmic}
\usepackage{color}
\usepackage[normalem]{ulem}
\usepackage{cancel}
\usepackage{amssymb}
\usepackage{color}

\usepackage[ruled,vlined,titlenotnumbered]{algorithm2e} 
\usepackage{cite}
\usepackage{float}

\newcommand{\R}{\mathbb{R}}
\newcommand{\xset}{\mathcal{X}}
\newcommand{\yset}{\mathcal{Y}}
\newcommand{\xfset}{\mathbb{X}}
\newcommand{\yfset}{\mathbb{Y}}

\newcommand{\reachset}{\mathcal{V}}
\newcommand{\targetset}{\mathcal{L}}
\newcommand{\traj}{\zeta} % trajectory


\newcommand{\pcset}{\mathcal{U}_p} %planner control set
\newcommand{\pcfset}{\mathbb{U}_p} %planner control function set
\newcommand{\tcset}{\mathcal{U}_s} %tracker control set
\newcommand{\tcfset}{\mathbb{U}_s} %tracker control funciton set
\newcommand{\dset}{\mathcal{D}}
\newcommand{\dfset}{\mathbb{D}}

\newcommand{\pset}{\mathcal{P}} %planner set set
\newcommand{\tset}{\mathcal{S}} %tracker set
\newcommand{\rset}{\mathcal{R}}

\newcommand{\tvar}{t}
\newcommand{\thor}{T} % Time horizon

\newcommand{\tstate}{s} % Tracker state
\newcommand{\pstate}{p} % Planner state
\newcommand{\rstate}{r} % Relative state


\newcommand{\ttraj}{\xi_{\tdyn}} % Tracker trajectory
\newcommand{\ptraj}{\xi_{\pdyn}} %Planner trajectory
\newcommand{\rtraj}{\xi_\rdyn}

\newcommand{\senseDist}{m}


\newcommand{\tctrl}{u_s} % Tracker control
\newcommand{\dstb}{d} % Disturbance
\newcommand{\pctrl}{u_p} % Planner control

\newcommand{\tdyn}{f} % Tracker dynamics
\newcommand{\pdyn}{h} % Planner Dynamics
\newcommand{\rdyn}{g} % Relative dynamics

\newcommand{\plannerfunc}{j}

\newcommand{\ptind}{i} % Index of vehicle state corresponding to planner state
\newcommand{\ptmat}{Q} % Matrix for transforming planner state to the same length as tracker state
\newcommand{\tpmat}{Q^T}

\newcommand{\errfunc}{l} % Error function
\newcommand{\valfunc}{V} % Value function

\newcommand{\deriv}{\nabla\valfunc} %gradient look-up table

\newcommand{\dx}{\Delta x} %distance allowed in a time step
\newcommand{\dt}{\Delta t} %time step

\newcommand{\obsSense}{\mathcal{O}_{sense}}
\newcommand{\obsAug}{\mathcal{O}_{aug}}



\newcommand{\TEB}{\mathcal B} % tracking error bound

\newtheorem{thm}{Theorem}
\newtheorem{claim}{Claim}
\newtheorem{rem}{Remark}
\newtheorem{prop}{Proposition}
\newtheorem{proof}{IEEEproof}

\newcommand{\MCnote}{\textcolor{blue}}
\newcommand{\SHnote}{\textcolor{red}}

\title{\LARGE \bf FaSTrack Journal Version}

\author{}
%\author{Sylvia L. Herbert*, Mo Chen*, SooJean Han, Somil Bansal, Jaime F. Fisac, and Claire J. Tomlin
%\thanks{This research is supported by ONR under the Embedded Humans MURI (N00014-16-1-2206). The research of S. Herbert has received funding from the NSF GRFP and the UC Berkeley Chancellor's Fellowship Program.}
%\thanks{* Both authors contributed equally to this work. All authors are with the Department of Electrical Engineering and Computer Sciences, University of California, Berkeley. \{sylvia.herbert, mochen72, soojean, somil, jfisac, tomlin\}@berkeley.edu}}


\begin{document}
\maketitle
\thispagestyle{empty}
\pagestyle{empty}

%%%
\begin{abstract}
Real-time and safe trajectory planning in unknown environments is vital to many applications of autonomous systems. Real-time trajectory planning typically requires simplified system dynamics planning at best, while safe trajectory planning tends to be computationally intensive. We propose FaSTrack, Fast and Safe Tracking, . A path or trajectory planner using simplified dynamics to plan quickly can be incorporated into the FaSTrack framework, which provides a safety controller for the vehicle along with a guaranteed tracking error bound. This bound captures all possible deviations due to high dimensional dynamics and external disturbances. Note that FaSTrack is modular and can be used with most current path or trajectory planners. We demonstrate this framework using a 10D nonlinear quadrotor model tracking a 3D path obtained from an RRT planner.
\end{abstract}

% !TEX root = tracking.tex
\section{Introduction}

Autonomous systems have a great potential to improve many industries in the near future. However, to achieve their potential there is a need to ensure the ability to make real-time plans while maintaining safety guarantees.
% As unmanned aerial vehicles (UAVs) and other autonomous systems become more commonplace, it is essential that they be able to plan safe motion paths through crowded environments in real-time. 
This is particularly crucial for navigating through environments that are \textit{a priori} unknown, because replanning based on updated information about the environment is often necessary. 
Achieving safe navigation in real time is difficult for many common dynamical systems due to the computational complexity of generating and formally verifying the safety of dynamically feasible trajectories.
 In order to achieve real-time planning, many algorithms use highly simplified model dynamics or kinematics to create a nominal trajectory that is then tracked by the system using a feedback controller such as a linear quadratic regulator (LQR).  These nominal trajectories may not be dynamically feasible for the true autonomous system, resulting in a tracking error between the planned path and the executed trajectory.
 This concept is illustrated in Fig. \ref{fig:chasing}, where the path was planned using a simplified planning model, but the real dynamical system cannot track this path exactly. 
Additionally, external disturbances (e.g. wind) can be difficult to account for using real-time path or trajectory planning algorithms, causing another source of tracking error. 
These tracking errors can lead to dangerous situations in which the planned path is safe, but the actual system trajectory enters unsafe regions.  Therefore, real-time planning is achieved at the cost of guaranteeing safety.  Common practice techniques augment obstacles by an ad hoc safety margin (see Fig. \ref{fig:chasing}, right), which may alleviate the problem but is performed heuristically and therefore does not guarantee safety.
 \begin{figure}
 	\centering
 	\includegraphics[width=0.35\textwidth]{fig/chasing}
 	\caption{Left: A planning algorithm uses a fast but simple model (blue disk), to plan around obstacles (gray disks). The more complicated tracking model (green plane) tracks the path. By using FaSTrack the autonomous system is guaranteed to stay within some TEB (black circle). Right: Safety can be guaranteed by planning with respect to obstacles augmented by the TEB (large black circles).}
 	\label{fig:chasing}
 \end{figure}
 %Real-time planning that is both safe and accurate presents a very difficult challenge: accuracy and robustness in many dyanimcal systme sis difficult to compute, often precluding real-time computer hands.fast planning is generally at odds with the need for maintaining safety and robustness.  

To attain fast planning speed while maintaining safety, we propose the modular framework FaSTrack: Fast and Safe Tracking.  As before, FaSTrack allows path or trajectory planning algorithms to use a simplified model of the system in order to operate in real time using augmented obstacles.  However, the bound for augmenting obstacles is rigorously computed and comes with a corresponding optimal tracking controller. Together this bound and controller guarantees safety for the autonomous system as it tracks the simplified plans.

We compute this bound and controller by modeling the navigation task as a pursuit-evasion game between a sophisticated \textit{tracking model} (pusuer) and the simplified \textit{planning model} of the system (evader). 
The tracking model accounts for complex system dynamics as well as bounded external disturbances, while the simple planning model enables the use of real-time planning algorithms. 
Offline, the pursuit-evasion game between the two models can be analyzed using any suitable method. 
This results in a \textit{tracking error function} that maps the initial relative state between the two models to the \textit{tracking error bound} (TEB): the maximum possible relative distance that could occur over time. 
This TEB can be thought of as a ``safety bubble" around the planning model of the system that the tracking model of the system is guaranteed to stay within.

When this precomputation converges, an invariant TEB can be computed for all time.  Since the planning model can be designed by the user, typically one can select a model such that the computation converges.  However, there may be cases in which convergence doesn't occur (i.e. even when acting optimally the autonomous sytem cannot keep up with the planning model used by the path or trajectory planning algorithm).  In these cases we can instead compute a time-varying TEB.  Intuitively, this means that as time progresses the tracking error bound increases by a known amount.

Because the tracking error is bounded in the relative state space, we can precompute and store the \textit{optimal tracking controller} that  maps the real-time relative state to the optimal tracking control for the tracking model of the sytstem to pursue the planning model of the system. 
The offline computations are \textit{independent} of the path planned in real time.

Online, the autonomous system senses local obstacles, which are then augmented by the TEB to ensure that no potentially unsafe paths can be computed. 
Next, any chosen path or trajectory planning algorithm uses the simplified planning model and the local environment to determine the next desired state. 
The autonomous system (represented by the tracking model) then finds the relative state between itself and the next desired state. 
If this relative state is nearing the TEB then it is plugged into the optimal tracking controller to find the instantaneous optimal tracking control of the tracking model required to stay within the error bound; otherwise, any tracking controller may be used. In this sense, FaSTrack provides a \emph{least-restrictive} control law.
This process is repeated as long as desired. 

FaSTrack was designed to be modular, and can be used with any method for computing the TEB in conjunction with any existing fast path or trajectory planning algorithms.  
This enables motion planning that is real-time, guaranteed safe, and dynamically accurate. 
In this paper, we demonstrate the FaSTrack framework by using three different real-time planning algorithms that have been ``robustified" by precomputing the TEB and tracking controller. 
The planning algorithms used in our numerical examples are the fast sweeping method (FSM) \cite{Takei2013}, rapidly-exploring random trees (RRT) \cite{Kuffner2000,Kavraki1996}, and model-predictive control (MPC) \cite{Qin2003}. 
In the three examples, we also consider different tracking and planning models.
The precomputation of the TEB and optimal tracking control function for each planning-tracking model pair is done by solving a Hamilton-Jacobi (HJ) partial differential equation (PDE). 
Two of the precomputations converge to an invariant TEB, and one uses a time-varying TEB.
In the simulations, the system travels through a static environment with constraints defined, for example, by obstacles, while experiencing disturbances.
The constraints are only fully known through online sensing, for example, once obstacles are within the limited sensing region of the autonomous system. 
By combining the TEB with real-time planning algorithms, the system is able to safely plan and track a trajectory through the environment in real time. 
% Introduction (.5-1p)
%%Tracking with quadrotors is a need
%%There exist methods that work in real time and methods that work for safety but not very many for both
%%Goal: combine both in a simple way

% !TEX root = tracking.tex
\section{Related Work \label{sec:relatedwork}}
\textcolor{red}{work on fast planning\\
work on safe planning\\
work on both\\
how ours is different}
% Related Work (1p)
%%work on fast planning
%%work on safe planning
%%work on both
%%how ours is different

% !TEX root = tracking.tex
\section{Problem Formulation \label{sec:formulation}}
In this paper we seek to simultaneously plan and track a trajectory (or path converted to a trajectory) online and in real time. 
The planning is done using a relatively simple model of the system, called the planning model. 
The tracking is done by a tracking model representing the autonomous system. 
The environment may contain static obstacles that are \textit{a priori} unknown and can be observed by the system within a limited sensing range (see Section \ref{sec:online}). 
In this section we define the tracking and planning models, as well as the goals of the paper.

\subsection{Tracking  Model}
The tracking model is a more accurate and typically higher-dimensional representation of the autonomous system dynamics than the planning model presented in Section \ref{sec:planning_model}. 
Let $\tstate$ represent the states of the tracking model. 
The evolution of the tracking model dynamics satisfy ordinary differential equation (ODE)

\begin{equation}
\begin{aligned}
\label{eq:tdyn}
\frac{d\tstate}{d\tvar} = \dot{\tstate} = \tdyn(\tstate(\tvar), \tctrl(\tvar), \dstb(\tvar)), \tvar \in [0, \thor], \\
\tstate(\tvar) \in \tset, \tctrl(\tvar) \in \tcset, \dstb(\tvar) \in \dset.
\end{aligned}
\end{equation}

We assume that the tracking model dynamics $\tdyn : \tset\ \times\ \tcset \times \dset \rightarrow \tset$ are uniformly continuous, bounded, and Lipschitz continuous in the system state $\tstate$ for a fixed control and disturbance functions $\tctrl(\cdot), \dstb(\cdot)$. The control function $\tctrl(\cdot)$ and disturbance function $\dstb(\cdot)$ are drawn from the following sets:

\begin{equation}
\begin{aligned}
\tctrl(\cdot) \in \tcfset(t) = \{\phi: [0, \thor] \rightarrow \tcset: \phi(\cdot) \text{ is measurable}\}\\
\dstb(\cdot) \in \dfset(t) = \{\phi: [0, \thor] \rightarrow \dset: \phi(\cdot) \text{ is measurable}\}
\end{aligned}
\end{equation}

\noindent where $\tcset, \dset$ are compact and $t\in[0, \thor]$ for some $T>0$. Under these assumptions there exists a unique trajectory solving (\ref{eq:tdyn}) for a given $\tctrl(\cdot) \in \tcfset, \dstb(\cdot)\in\dfset$ \cite{Coddington84}. The trajectories of (\ref{eq:tdyn}) that solve this ODE will be denoted as $\ttraj(\tvar; \tstate, \tvar_0, \tctrl(\cdot), \dstb(\cdot))$, where $\tvar_0,\tvar \in [0, \thor]$ and $\tvar_0 \leq \tvar$. This trajectory notation represents the state of the system at time $\tvar$, given that the trajectory is initiated at state $\tstate$ and time $\tvar_0$ and applied control signal $\tctrl(\cdot)$ and disturbance signal $\dstb(\cdot)$.  These trajectories will satisfy the initial condition and the ODE (\ref{eq:tdyn}) almost everywhere:

\begin{align*}
&\frac{d}{d\tvar}\ttraj(\tvar; \tstate_0, \tvar_0, \tctrl(\cdot), \dstb(\cdot)) = \\ &\qquad \tdyn(\ttraj(\tvar; \tstate_0, \tvar_0, \tctrl(\cdot), \dstb(\cdot)), \tctrl(\tvar), \dstb(\cdot)), \\
&\ttraj(\tvar_0; \tstate_0, \tvar_0, \tctrl(\cdot), \dstb(\cdot)) = \tstate_0.
\end{align*}

Let $\goal \subset \tset$ represent the set of goal states, and let $\tconstr \subset \tset$ represent state constraints that must be satisfied for all time.
Often, $\tconstr$ represents the complement of obstacles that the system must avoid.

\subsection{Planning Model \label{sec:planning_model}}
The planning model is a simpler, lower-dimensional model of the system.
For navigation in unknown environments, fast replanning is necessary, so the planning model is typically one that allows the desired planning algorithm to operate in real time.
%For examples of planning models, see Section \ref{sec:results}.

Let $\pstate$ represent the state variables of the planning model, with control $\pctrl$. 
We assume that the planning states $\pstate \in \pset$ are a subset of the tracking states $\tstate \in \tset$, so that $\pset$ is a subspace within $\tset$.
This assumption is reasonable since a lower-fidelity model of a system typically involves a subset of the system's states, as with the numerical examples provided in this paper.
The dynamics of the planning model satisfy the ODE

\begin{align}
\label{eq:pdyn}
\frac{d\pstate}{d\tvar} = \dot{\pstate} = \pdyn(\pstate, \pctrl), \tvar \in [0, \thor], \quad \pstate \in \pset, \pctrl \in \pcset
\end{align}

\noindent with the analogous assumptions on continuity and boundedness as those for \eqref{eq:tdyn}.

Note that the planning model does not include a disturbance input. 
This is a key feature of FaSTrack: the treatment of disturbances is only necessary in the tracking model, which is modular with respect to any planning method. Therefore we can and will assume that the planning model (and the planning algorithm) do not consider disturbances.

Let $\goal \subset \pset$ and $\constr \subset \pset$ denote the projection of $\tgoal$ and $\tconstr$ respectively onto the subspace $\pset$.
We will assume that $\constr$ is \textit{a priori} unknown, and must be sensed as the autonomous system moves around in the environment.
Therefore, for convenience, we denote the currently known, or ``sensed'' constraints as $\constrSense(t)$.
Note that $\constrSense(t)$ depends on time, since the system may gather more information about, for example, obstacles over time.
In addition, as described throughout the paper, we will augment $\constrSense(t)$ according to the TEB between the tracking and planning models.
We denote the augmented obstacles as $\constrAug(t)$.

\subsection{Goals and Approach}
Given system dynamics in \eqref{eq:tdyn}, initial state $\tstate_0$, goal states $\goal$, and constraints $\tconstr$ such that $\constr$ is \textit{a priori} unknown and determined in real time, we would like to steer the system to $\goal$ with formally guaranteed satisfaction of $\tconstr$.

To achieve this goal, FaSTrack decouples the formal guarantee of safety from the planning algorithm.
Instead of having the system, represented by the tracking model, directly plan trajectories towards $\goal$, in our framework the autonomous system (represented by the tracking model) ``chases'' the planning model of the system, which uses any planning algorithm to obtain trajectories in real time.
When the planning model of the system reaches the goal set, the autonomous system will be contained within the goal set augmented by the TEB. The ensure that the autonomous sytem has in fact reached the goal, the planning model should end within the goal set contracted by the TEB.
Safety is formally guaranteed through precomputation of a TEB along with a corresponding optimal tracking controller, in combination with augmentation of constraints based on this TEB.
An illustration of our framework applied to a navigation task is shown in Figure \ref{fig:chasing}.
% formally introduce the problem

% !TEX root = tracking.tex
\section{General Framework \label{sec:framework}}
Given a dynamical system, we propose a hierarchical framework for combining Hamilton-Jacobi safety analysis with planning methods in a modular way.

Given: Target, mechanism for sensing obstacles

Goal: reach target without colliding with obstacles

Offline: compute bubble and error-feedback controller (tracker)

Online: At every time iteration,
\begin{enumerate}
  \item Sense and update obstacles (can also be done every N iterations)
  \item Augment obstacles according to bubble
  \item Plan path or trajectory using planner, assuming currently sensed and augmented obstacles
  \item Robustly track trajectory using tracker
\end{enumerate}


\textbf{Maybe put next paragraph in the introduction}

There are many fast planners that could potentially do planning in real-time; however, these typically cannot account for disturbances in a provably safe way. In addition, complex system models with nonlinear dynamics complicate planning algorithms (non-convex for MPC, more difficult for RRT). On the other hand, HJ reachability is able to handle disturbances, and is agnostic to system dynamics. In addition, provably guarantees can be provided. However, HJ reachability and in general formal verification methods can be very expensive to compute.

Refer to figure: planning level and safety level. 

In the safety level, we start with the error dynamics, and we compute two things: bubble which is fed into planner to plan with extra margin, and error-feedback controller for real-time control. These two can be computed offline independent of the planned path.

In the planning level, any planning method such as MPC, RRT, etc. (cite some things) can be used. The planning level does not need to take into account disturbances, and can use simple system dynamics or even no dynamics at all. In fact we will be using a simple RRT planner which simply provides paths, in the form of a sequence of line segments, which are not dynamically feasible. 

\begin{figure}
\includegraphics[width=\columnwidth]{fig/framework_online}
\caption{Online framework}
\label{fig:fw_online}
\end{figure}

\begin{figure}
  \includegraphics[width=\columnwidth]{fig/framework_offline}
  \caption{Offline framework}
  \label{fig:fw_offline}
\end{figure}

\begin{figure}
  \includegraphics[width=\columnwidth]{fig/hybrid_controller}
  \caption{Hybrid controller}
  \label{fig:hybrid_ctrl}
\end{figure}
%framework of algorithm

% !TEX root = tracking.tex
\section{Offline Computations \label{sec:precomp}}
The offline computation begins with setting up a capture-avoid game between the tracking system and the planning system, which we then analyze using Hamilton Jacobi (HJ) reachability. In this game, the tracking system will try to "capture" the planning system, while the planning system is doing everything it can to avoid capture. In reality the planner is typically not actively be trying to avoid the tracking system, but this allows us to account for worst-case scenarios. If both systems are acting optimally in this way, we want to determine the largest relative distance that may occur over time. This distance is the maximum possible tracking error between the two systems. Note that this tracking error is independent of specific online paths; it depends only on relative states and dynamics.

\subsection{Relative Dynamics}
To determine the relative distance that may occur over time we must first define the relative states and dynamics between the tracking and planning models. The individual dynamics are defined in Section \ref{sec:formulation}, equations (\ref{eq:tdyn}) and (\ref{eq:pdyn}). The relative system is found by fixing the planning model to the origin and finding the states of the tracking model relative to the planning model, as shown below.

\begin{equation}
\label{eq:rdyn}
\begin{aligned}
\rstate = \tstate - \ptmat\pstate\\
\dot\rstate = \rdyn(\rstate, \tctrl, \pctrl, \dstb)
\end{aligned}
\end{equation}

Matrix $\ptmat$ matches the common states of $\tstate$ and $\pstate$ by augmenting the state space of the planning model. The relative states $\rstate$ now represent the tracking states relative to the planning states. A similar matrix $\tpmat$ projects the state space of the tracking model onto the planning model: $\pstate = \tpmat(\tstate-\rstate)$. This will be used to update the planning model in the online algorithm.

\subsection{Formalizing the Capture-Avoid Game}
Now that we have the relative dynamics between the two systems we must define a metric for the tracking error bound between these systems. We do this by defining an implicit surface function as a cost function $\errfunc(\rstate)$ in the new frame of reference. Because the metric we care about is distance to the origin (and thus distance to the planner), this cost function can be as simple as distance in position space to the origin. An example can be seen in Figure \ref{fig:quad4D_example}-a, where the contour rings represent varying level sets of the cost function. The tracking system will try to minimize this cost to reduce the relative distance, while the planning system will do the opposite.

\begin{figure}
	\centering
	\includegraphics[width=0.5\textwidth]{fig/quad4D_example}
	\caption{illustrative example of the precomputation steps for a 4D quadrotor tracking a 2D kinematic planner. All graphs are defined over a 2D slice of the 4D system. a) Cost function $\errfunc(\rstate)$ defined on relative states as distance to the origin, b) Value function $\valfunc(\rstate)$ computed using HJ reachability, c) Level sets of $\errfunc(\rstate)$ (solid) and $\valfunc(\rstate)$ (dashed). If the initial relative state is contained within the dashed set the system is guaranteed to remain within the solid set (i.e. the tracking error bound)}
	\label{fig:quad4D_example}
\end{figure} 
 
 We want to find the farthest distance (and thus highest cost) that this game will ever reach when both players are acting optimally. Therefore we want to find a mapping between the initial relative state of the system and the maximum possible cost achieved over the time horizon. This mapping is through our value function, defined as: \MCnote{Time horizon, time variable notation, control/disturbance functions, trajectory notation}
 \begin{equation}
 \begin{aligned}
 \label{eq:valfunc}
 	V(\rstate)= \sup_{\pctrl(\cdot)} \inf_{\tctrl(\cdot), \dstb(\cdot)} \big\{\sup_{\tvar\in [0, \thor]} \errfunc(\rtraj(\tvar; \rstate, 0, \pctrl(\cdot), \tctrl(\cdot)))\big\}
 	\end{aligned}
 \end{equation} 
 
 This is a modified version of the Hamilton-Jacobi formulation as described by \cite{Fisac15}. By implementing HJ reachability analysis we solve for this value function over the time horizon. If the control authority of the tracking system is powerful enough to always eventually reach the planning system, this value function will converge to an invariant solution for all time. An example of this value function is in Figure \ref{fig:quad4D_example}-b. In the next section we will prove that the sub-level sets of this value function will map initial relative states to the guaranteed furthest possible tracking error over all time, as seen in Figure \ref{fig:quad4D_example}-c.
 
 \SHnote{In preparation for the online computation described in Section \ref{sec:online}, the value function $\valfunc(\rstate)$ is saved over a grid representing the relative state space. Similarly the spatial gradients of the value function $\nabla\valfunc(\rstate)$ are saved as the safety control look-up table $\deriv$}.
 
\subsection{Main Result}
 \begin{equation}
 \begin{aligned}
& \valfunc(\rstate, \thor) = \inf_{\pctrl(\cdot)} \sup_{\pctrl(\cdot), \dstb(\cdot)} \big\{\\
&\quad \max_{\tvar \in [0, \thor]} \errfunc(\rtraj(\tvar; \rstate, 0, \pctrl(\cdot), \pctrl(\cdot), \dstb(\cdot)))\big\} \\
& \valfunc(\rstate, \thor) = \max_{\tvar \in [0, \thor]} \errfunc(\rtraj^*(\tvar; \rstate, 0)) 
 \end{aligned}
  \end{equation}
 
 \begin{claim}
   \label{thm:main}
   Let $\thor_c \ge 0$, and suppose
   
   \begin{equation}
   \label{eq:conv_valfunc}
   \valfunc_\infty(\rstate) := \valfunc(\rstate, \thor) = \valfunc(\rstate, \thor_c) ~ \forall \thor \ge \thor_c.
   \end{equation}
   
   Then $\forall \tvar_1, \tvar_2$ with $\tvar_2 \ge \tvar_1$,
   
   \begin{equation}
   \label{eq:invariant}
   \valfunc_\infty(\rstate) \ge \valfunc_\infty\Big(\rtraj^*(\tvar_2;, \rstate, \tvar_1)\Big)
   \end{equation}
   
   \noindent where
   \begin{equation}
   \begin{aligned}
   & \rtraj^*(0;, \rstate, \tvar) := \rtraj(0; \rstate, \tvar, \pctrl^*(\cdot), \pctrl^*(\cdot), \dstb^*(\cdot))) \\
   & \pctrl^*(\cdot) = \arg \inf_{\pctrl(\cdot)} \sup_{\pctrl(\cdot), \dstb(\cdot)}\big\{ \\
   & \qquad \max_{\tvar \in [0, \thor]} \errfunc(\rtraj(0; \rstate, \tvar, \pctrl(\cdot), \pctrl(\cdot), \dstb(\cdot))) \big\}\\
   & \pctrl^*(\cdot) = \arg \sup_{\pctrl(\cdot)} \sup_{\dstb(\cdot)} \big\{ \\
   & \qquad \max_{t \in [0, \thor]} \errfunc(\rtraj(0; \rstate, \tvar, \pctrl(\cdot), \pctrl(\cdot), \dstb(\cdot))) \big\} \\
   & \dstb^*(\cdot) = \arg \sup_{\dstb(\cdot)} \big\{\max_{\tvar \in [0, \thor]} \errfunc(\rtraj(0; \rstate, \tvar, \pctrl(\cdot), \pctrl(\cdot),  \dstb(\cdot))) \big\}
   \end{aligned}
   \end{equation}
   
 \end{claim}
 
\textit{Proof:}

Assume $\thor \ge \thor_c$.

\begin{equation}
\begin{aligned}
\valfunc(\rstate, \thor) &= \valfunc_\infty(\rstate) = \max_{\tvar \in [0, \thor]} \errfunc(\rtraj^*(\tvar; \rstate, 0))\\
\end{aligned}
\end{equation}

By time-invariance
 \begin{equation}
 \begin{aligned}
 \valfunc_\infty(\rstate) &= \max_{\tvar \in [0, \thor]} \errfunc(\rtraj^*(\tvar; \rstate, 0)) \\
 &= \max_{\tvar \in [\thor_c-\thor, \thor_c]} \errfunc(\rtraj^*(\tvar; \rstate, \thor_c-\thor)) \\
 \end{aligned}
 \end{equation}  
 
Consider time subinterval:
 
 \begin{equation}
\begin{aligned}
\valfunc_\infty(\rstate) &= \max_{\tvar \in [0, \thor]} \errfunc(\rtraj^*(\tvar; \rstate, 0)) \\
&\ge \max_{\tvar \in [0, \thor_c]} \errfunc(\rtraj^*(\tvar; \rstate, \thor_c-\thor)) \\
\end{aligned}
\end{equation}  

Now look at trajectory $\rtraj^*(\tvar; \rstate, \thor_c-\thor)$:

\begin{equation}
\begin{aligned}
\rtraj^*(\tvar; \rstate, \thor_c-\thor) = \rtraj^*(\tvar; \rtraj^*(0; \rstate, \thor_c-\thor), 0)
\end{aligned}
\end{equation}

Continuing from value function:

\begin{equation}
\begin{aligned}
\valfunc_\infty(\rstate) &\ge \max_{\tvar \in [0, \thor_c]} \errfunc(\rtraj^*(\tvar; \rtraj^*(0; \rstate, \thor_c-\thor), 0)) \\
&= \valfunc_\infty(\rtraj^*(0; \rstate, \thor_c-\thor))
\end{aligned}
\end{equation} 
 
 By time-invariance again,
 
 \begin{equation}
 \begin{aligned}
 \valfunc_\infty(\rstate) &\ge \valfunc_\infty(\rtraj^*(\thor-\thor_c; \rstate, 0)) ~ \forall \thor\ge\thor_c \\
 \Leftrightarrow  \valfunc_\infty(\rstate) &\ge \valfunc_\infty(\rtraj^*(\tvar_2; \rstate, 0)) ~ \forall \tvar_2 \ge 0
 \end{aligned}
 \end{equation} 
 
   Since the system dynamics are time-invariant, we can pick $\tvar_2 = -\thor_c$ without loss of generality, and $\tvar_1 = -\thor$ to obtain the desired result. \hfill $\blacksquare$
 
\begin{rem}
  The interpretation of Theorem \ref{thm:main}, particularly \eqref{eq:invariant}, is that every level set of $\valfunc_\infty(\rstate)$ is invariant under the following conditions:
  \begin{itemize}
    \item The real system applies the optimal control which tries to track the virtual system;
    \item The virtual system applies the optimal virtual control that tries to escape from the real system;
    \item The real system experiences the worst-case disturbance.
  \end{itemize}
  
  In particular, the value of $\underline\valfunc := \valfunc_\infty(\rstate)$ can be interpreted as the smallest possible tracking error \MCnote{(we need to explicitly define tracking error)} given the above assumptions. The tracking error bound in Fig. \ref{fig:fw_online}, \ref{fig:hybrid_ctrl}, \ref{fig:fw_offline} is given by\footnote{In practice, since $\valfunc_\infty$ is obtained numerically, we set $\TEB = \{\rstate: \valfunc_\infty(\rstate) \le \underline\valfunc + \epsilon\}$ for some suitably small $\epsilon>0$} the set $\TEB = \{\rstate: \valfunc_\infty(\rstate) \le \underline\valfunc\}$. \SHnote{mention that the tracking bound in the planning reference frame is:
  	\begin{equation} \label{eq:TEBp}
  	\TEB_\pstate(\tstate) = \{\pstate: \valfunc_\infty(\tstate-\ptmat\pstate) \le \underline\valfunc\}
  	\end{equation}
  	and this is what we will use for augmenting obstacles}
  
  \MCnote{what if disturbances are not optimal?}
\end{rem}
 
 
 \begin{rem} 
   Theorem \ref{thm:main} is very similar to well-known results in differential game theory with a slightly different cost function \cite{}, and has been utilized in the context of using the subzero level set of $\valfunc_\infty$ as a backward reachable set for tasks such as collision avoidance or reach-avoid games \cite{}. In our work we do not assign special meaning to any particular level set, and instead consider all level sets at the same time. This effectively allows us to perform solve many simultaneous reachability problems in a single computation, thereby removing the need to check whether resulting invariant sets are empty, as was previously done in \MCnote{SPP paper \cite{}}.
 \end{rem}

\SHnote{say that this implicitly encodes all BRSs of initial states from which the tracking error reaches each level curve of the implicit value function}

% Computing capture basin (~2.5p)
%% HJ Reachability (~1p)
%% Relative dynamics, setup, etc. (~1p)
%% Capture basin computation (~0.5p)

% !TEX root = tracking.tex
\section{Online Computation \label{sec:online}}

\MCnote{Sampled data}

The following list describes the online computation. Be sure to begin your computation by starting the path planner at the same position as the true system. This will allow for the smallest tracking error bound.
\SHnote{details of size of sensing radius needed, dt needed, dx needed, how finding optimal control works}

\begin{algorithm}[bp]
	
	\caption{Online Trajectory Planning}
	\label{alg:algOnline}
	\begin{algorithmic}[1]
		\STATE \textbf{Initialization}: \label{ln:Istart}
		\STATE Inputs: tracking error look-up table $\valfunc$, optimal control look-up table $\deriv$
		\STATE $\pstate = \tstate = 0$
		\STATE $\rstate = \tstate - \ptmat\pstate$
		\STATE $\TEB = \valfunc(\rstate)$
		\STATE $\senseDist = 2\TEB+\dx$
		
		\WHILE{planning goal is not reached}
		\STATE \textbf{Obstacles Block}:
		\STATE Sense obstacles within sensing range $\senseDist$
		\STATE Expand sensed obstacles by $\TEB$
		
		\STATE \textbf{Path Planner Block}:
		\STATE Input state $\pstate$ and augmented obstacles to path planner; output $\pstate_{new}$
		
		\STATE \textbf{Hybrid Tracking Controller Block}:
		\STATE $\rstate_{next} = \tstate - \ptmat\pstate_{next}$
		
		\IF{$\rstate_{next}$ is near boundary $\TEB$} 
		\STATE {use safety controller:}
		\STATE {$\tctrl = \arg\min_{\tctrl}<\rdyn(\rstate_{next},\tctrl,\pctrl,\dstb),deriv\{\rstate_{next}\}>$}
		\ELSE \STATE{use performance controller} \ENDIF
		
		\STATE \textbf{Tracking Model Block}:
		\STATE apply control $\tctrl$ to vehicle for a time step of $\dt$, save next state $\tstate_{next}$
		
		\STATE \textbf{Planning Model Block}:
		\STATE $\pstate = \ptmat^\intercal\tstate_{next}^\intercal$
		\STATE check if $\pstate$ is at planning goal
		\STATE reset states $\tstate = \tstate_{next}, \rstate = \rstate_{next}$
		\ENDWHILE
	\end{algorithmic}
\end{algorithm}


%\begin{enumerate}
%\item Tracking Error Block
%\begin{itemize}
%	\item Inputs: sensor information, tracking error bound
%	\item Output: augmented obstacles
%	\item Sense local environment, locate obstacles
%	\item Expand sensed obstacles by the tracking error bound
%\end{itemize}
%\item Path Planner Block
%\begin{itemize}
%	\item Inputs: virtual  state, augmented obstacles
%	\item Output: desired next virtual state
%	\item Use current virtual state and augmented obstacles to find desired next state
%\end{itemize}
%\item Hybrid Tracking Controller
%\begin{itemize}
%	\item Inputs: desired next virtual state, true model state
%	\item Output: true model control
%	\item Compute relative state
%	\item Compare relative state to tracking error bound. If the relative state is greater than dx from the tracking error bound, use the performance controller. Otherwise, use the safety controller.
%	\item Performance Controller: \SHnote{performance controller}
%	\item Safety Controller: Use controller lookup table to find the spatial gradients of the value function at that relative state. Plug in spatial gradients to the hamiltonian; find argmin control
%\end{itemize}
%\item True System Block
%\begin{itemize}
%	\item Input: control, current true system state
%	\item Output: updated true system state
%	\item Propagate true system by dt using given control
%\end{itemize}
%\item Virtual System Block
%\begin{itemize}
%	\item Input: true system state
%	\item Output: virtual system state
%	\item Project true system state onto virtual system subspace; use this as the next virtual system state
%	\item Check if goal has been reached.  If not, repeat loop
%\end{itemize}
%\end{enumerate}
% online part of framework

% !TEX root = tracking.tex
\section{5D Car Reachability Example \label{sec:reach_planner}}
\MCnote{5D Car tracking Dubins car.}

\MCnote{Jaime is computing tracking reachable set via GPU}

\MCnote{Mo is implementing 3D Dubins Car real-time tracking with fast sweeping method.}

\MCnote{(Actually, FSM is actually not the same as time-dependent reachability)}

% !TEX root = tracking.tex
\subsection{10D quadrotor-3D single integrator example with RRT\label{sec:resultsRRT}}

\MCnote{Convergence (2 slices for 4D, 2D, 3D space)}

\begin{figure*}
	\centering
	\includegraphics[width=0.7\textwidth]{fig/quad10D_example_cost}
	\caption{On the left are the cost and value functions over a 2D slice of the 10D relative state space, with contour lines showing three level sets of these functions. On the right are 3D projections of these level sets at the same slice $(v_{x},v_{y},v_{z})=[1, -1, 1]$ m/s, $(\theta_{x},\omega_{x},\theta_{y},\omega_{y})=0$. The solid boxes show initial relative states, and the transparent boxes show the corresponding tracking error bound. In practice we set the initial relative states to 0 to find the smallest invariant tracking error bound.}
	\label{fig:quad10D_example}
	\end{figure*} 
\begin{figure}
	\centering
	\includegraphics[width=0.25\textwidth]{fig/quad10D_slices}
	\caption{Various 3D slices of the 10D relative states (solid) and the corresponding tracking error bound (transparent)}
	\label{fig:quad10D_example_slices}
\end{figure} 

Our second example involves a 10D near-hover quadrotor developed in \cite{Bouffard12} as the tracking model and a single integrator in 3D space as planning model.
Planning is done using RRT, a well-known sampling-based planner that quickly produces geometric paths from a starting position to a goal position \cite{Kuffner2000,Kavraki1996}.
Paths given by the RRT planner is converted to time-stamped trajectories by placing a maximum velocity in each dimension along the generated geometric paths.

The dynamics of tracking model and of the 3D single integrator is as follows:

\begin{equation}
\label{eq:Quad10D_dyn}
\begin{bmatrix}
\dot{x}\\
\dot{v_x}\\
\dot{\theta_x}\\
\dot\omega_x\\
\dot{y}\\
\dot{v_y}\\
\dot{\theta_y}\\
\dot\omega_y\\
\dot{z}\\
\dot{v_z}
\end{bmatrix}
=
\begin{bmatrix}
v_x + d_x\\
g \tan \theta_x\\
-d_1 \theta_x + \omega_x\\
-d_0 \theta_x + n_0 a_x\\
v_y + d_y\\
g \tan \theta_y\\
-d_1 \theta_y + \omega_y\\
-d_0 \theta_y + n_0 a_y\\
v_z + d_z\\
k_T a_z - g
\end{bmatrix}, \quad
\begin{bmatrix}
\dot{\hat x}\\
\dot{\hat y}\\
\dot{\hat z}\\
\end{bmatrix} =
\begin{bmatrix}
\hat v_x \\
\hat v_y \\
\hat v_z
\end{bmatrix}
\end{equation}
\noindent where quadrotor states $(x, y, z)$ denote the position, $(v_x, v_y, v_z)$ denote the velocity, $(\theta_x, \theta_y)$ denote the pitch and roll, and $(\omega_x, \omega_y)$ denote the pitch and roll rates. 
The controls of the 10D system are $(u_x, u_y, u_z)$, where $u_x$ and $u_y$ represent the desired pitch and roll angle, and $u_z$ represents the vertical thrust.

The 3D system controls are $(\hat v_x, \hat v_y, \hat v_z)$, and represent the velocity in each positional dimension. 
The disturbances in the 10D system $(\dstb_x, \dstb_y, \dstb_z)$ are caused by wind, which acts on the velocity in each dimension. 

The model parameters are chosen to be $d_0=10$, $d_1=8$, $n_0=10$, $k_T=0.91$, $g=9.81$, $|u_x| \le $, $|u_y| \le $, $u_z \in []$, $|\hat v_x|, |\hat v_y|, |\hat v_z| \le $.

\subsubsection{Offline computation}
We define the relative system states to consist of the error states, or relative position $(x_r, y_r, z_r)$, concatenated with the rest of the state variables of the 10D quadrotor model.
Defining $\rtrans = \mathbf I_{10}$ and 

\begin{equation*}
\ptmat = 
\begin{bmatrix}
  \begin{bmatrix} 1 \\ \mathbf 0_{3 \times 1} \end{bmatrix} 
    & \mathbf 0_{4\times 1} 
    & \mathbf 0_{4\times 1} \\
  \mathbf 0_{4\times 1} 
    & \begin{bmatrix} 1 \\ \mathbf 0_{3 \times 1} \end{bmatrix} 
    &  \mathbf 0_{4\times 1} \\
  \mathbf 0_{2\times 1} 
    & \mathbf 0_{2\times 1} 
    & \begin{bmatrix} 1 \\ 0 \end{bmatrix}
\end{bmatrix},
\end{equation*}

\noindent we obtain the following relative system dynamics:

\begin{equation}
\label{eq:Quad10DRel_dyn}
\begin{bmatrix}
\dot{x_r}\\
\dot{v_{x}}\\
\dot{\theta_{x}}\\
\dot\omega_{x}\\
\dot{y_r}\\
\dot{v_{y}}\\
\dot{\theta_{y}}\\
\dot\omega_{y}\\
\dot{z_r}\\
\dot{v_{z}}
\end{bmatrix} =
\begin{bmatrix}
v_x - \hat v_x + d_x\\
g \tan \theta_x\\
-d_1 \theta_x + \omega_x\\
-d_0 \theta_x + n_0 u_x\\
v_y - \hat v_y + d_y\\
g \tan \theta_y\\
-d_1 \theta_y + \omega_y\\
-d_0 \theta_y + n_0 u_y\\
v_z - \hat v_z + d_z\\
k_T u_z - g
\end{bmatrix}.
\end{equation}

Next, we follow the setup in section \ref{sec:precomp} to create a cost function, which we then evaluate using HJ reachability until convergence to produce the invariant value function as in (\ref{eq:valfunc}). Historically this 10D nonlinear relative system would be intractable for HJ reachability analysis, but using new methods in \cite{Chen2016DecouplingExact, Chen2016DecouplingJournal} we can decompose this system into 3 subsystems (for each positional dimension). Doing this also requires decomposing the cost function; therefore we represent the cost function as a 1-norm instead of a 2-norm. This cost function as well as the resulting value function can be seen projected onto the $x,y$ dimensions in Fig. \ref{fig:quad10D_example}.

Fig. \ref{fig:quad10D_example} also shows 3D positional projections of the mapping between initial relative state to maximum potential relative distance over all time (i.e. tracking error bound). If the real system starts exactly at the origin in relative coordinates, its tracking error bound will be a box of $\underline\valfunc = 0.81$ m in each direction. Slices of the 3D set and corresponding tracking error bounds are also shown in Fig. \ref{fig:quad10D_example_slices}. We save the look-up tables of the value function (i.e. the tracking error function) and its spatial gradients (i.e. the safety controller function).

\subsubsection{Online sensing and planning}
Our precomputed value function can serve as a tracking error bound, and its gradients form a look-up table for the optimal tracking controller. These can be combined with any planning algorithm such as MPC, RRT, or neural-network-based planners in a modular way. 

To demonstrate the combination of fast planning and provably robust tracking, we used a simple multi-tree RRT planner implemented in MATLAB modified from \cite{Gavin2013}. We assigned a speed of $0.5$ m/s to the piecewise linear paths obtained from the RRT planner, so that the planning model is as given in \eqref{eq:Quad10D_dyn}. Besides planning a path to the goal, the quadrotor must also sense obstacles in the vicinity. For illustration, we chose a simple virtual sensor that reveals obstacles within a range of 2 m in the $x$, $y$, or $z$ directions.

Once an obstacle is sensed, the RRT planner replans while taking into account all obstacles that have been sensed so far. To ensure that the quadrotor does not collide with the obstacles despite error in tracking, planning is done with respect to augmented obstacles that are ``expanded'' from the sensed obstacles by $\underline\valfunc$ in the $x$, $y$, and $z$ directions.

On an unoptimized MATLAB implementation on a desktop computer with a Core i7-2600K CPU, each iteration took approximately $25$ ms on average. Most of this time is spent on planning: obtaining the tracking controller took approximately $5$ ms per iteration on average. The frequency of control was once every $100$ ms.

Fig. \ref{fig:sim} shows the simulation results. Four time snapshots are shown. The initial position is $(-12, 0, 0)$, and the goal position is $(12, 0, 0)$. The top left subplot shows the entire trajectory from beginning to end. In all plots, a magenta star represents the position of the planning model; its movement is based on the paths planned by RRT, and is modeled by a 3D holonomic vehicle with a maximum speed. The blue box around the magenta star represents the tracking error bound.
\begin{figure}
	\centering
	\begin{subfigure}[t]{0.49\columnwidth} \label{subfig:sim_4}
		\includegraphics[width=\columnwidth]{fig/1173}
	\end{subfigure}  
	\begin{subfigure}[t]{0.49\columnwidth} \label{subfig:sim_1}
		\includegraphics[width=\columnwidth]{fig/224}
		\caption{}
	\end{subfigure}
	
	\begin{subfigure}[t]{0.49\columnwidth} \label{subfig:sim_2}
		\includegraphics[width=\columnwidth]{fig/763}
		\caption{}
	\end{subfigure}  
	\begin{subfigure}[t]{0.49\columnwidth} \label{subfig:sim_3}
		\includegraphics[width=\columnwidth]{fig/1042}
		\caption{}
	\end{subfigure}
	\caption{Numerical simulation. The tracking model trajectory is shown in blue, the planning model position in magenta, unseen obstacles in gray, and seen obstacles in red. The translucent blue box represents the tracking error bound. The top left subplot shows the entire trajectory; the other subplots zoom in on the positions marked in the top left subplot. The camera angle is also adjusted to illustrate our theoretical guarantees on tracking error and robustness in planning. A video of this simulation can be found at https://youtu.be/ZVvyeK-a62E \label{fig:sim}}
\end{figure}
The position of the tracking model is shown in blue. Throughout the simulation, the tracking model's position is always inside the tracking error, in agreement with Proposition \ref{prop:main}. In addition, the tracking error bound never intersects with the obstacles, a consequence of the RRT planner planning with respect to a set of augmented obstacles (not shown). In the latter two subplots, one can see that the quadrotor appears to be exploring the environment briefly before reaching the goal. In this paper, we did not employ any exploration algorithm; this exploration behavior is simply emerging from replanning using RRT whenever a new part (a $3$ m$^2$ portion) of an obstacle is sensed.

% !TEX root = tracking.tex
\subsection{8D Quadrotor MPC Example \label{sec:resultsMPC}}

In this section, we demonstrate the online computation framework in Algorithm~\ref{alg:algOnline} with an 8D quadrotor example. Unlike in Sections \ref{sec:reach_planner} and \ref{sec:resultsRRT}, we consider a time-varying TEB and utilize the Model Predictive Control (MPC) technique as the online planner. 

First we define the 8D dynamics of the tracking quadrotor, and the 4D dynamics of a double integrator, which serves as the planning system to be used in MPC:

\begin{equation}
\label{eq:Quad8D_dyn}
\begin{aligned}
\begin{array}{c}
\left[
\begin{array}{c}
\dot x\\
\dot v_x\\
\dot \theta_x\\
\dot \omega_x\\
\dot y\\
\dot v_y\\
\dot \theta_y\\
\dot \omega_y
\end{array}
\right]
=
\left[
\begin{array}{c}
v_{x,s} + d_x\\
g \tan \theta_x\\
-d_1 \theta_x + \omega_x\\
-d_0 \theta_x + n_0 a_x\\
v_y\\
g \tan \theta_y + \MCnote{d_{v_y}}\\
-d_1 \theta_y + \omega_y\\
-d_0 \theta_y + n_0 a_y
\end{array}
\right], \quad
\left[
\begin{array}{c}
\dot {\hat x}\\
\dot {\hat v}_x\\
\dot {\hat y}\\
\dot {\hat v}_y\\
\end{array}
\right] 
=
\left[
\begin{array}{c}
\hat v_x\\
\hat a_x\\
\hat v_y\\
\hat a_y\\
\end{array}
\right],
\end{array}\\
\end{aligned}
\end{equation}
where the states and controls of the 8D system are defined the same as \eqref{eq:Quad10D_dyn}. The position $(\hat x,\hat y)$ and velocity $(\hat v_x, \hat v_y)$ are the states of the 4D system. The controls are $(\hat a_x, \hat a_y)$, which represent the acceleration in each positional dimension. Note that the disturbances in the 8D system $(d_{v_x},d_{v_y})$ act on the \MCnote{acceleration} in each direction. \MCnote{***Explain where the disturbances come from (physically)***}

\subsubsection{Offline precomputation}

The relative dynamics between the 8D and the 4D system are defined using (\ref{eq:rdyn}):
%
% \begin{equation}
% \begin{aligned}
% \dot x_r &= \dot x - \dot{\hat x} = v_x - \hat v_x = v_{x,r}\\
% \dot v_{x,r} &= \dot v_x - \dot{\hat v}_x = g \tan \theta_x - \hat a_x  + \MCnote{d_{v_x}}\\
% \dot y_r &= \dot y - \dot{\hat y} = v_y - \hat v_y = v_{y,r}\\
% \dot v_{y,r} &= \dot v_y - \dot{\hat v}_y = g \tan \theta_y - \hat a_y  + \MCnote{d_{v_y}}\\
% \end{aligned}
% \end{equation}
%
\begin{equation}
\label{eq:Quad8DRel_dyn}
\begin{aligned}
\begin{array}{c}
\left[
\begin{array}{c}
\dot x_r\\
\dot v_{x,r}\\
\dot \theta_x\\
\dot \omega_x\\
\dot y_r\\
\dot v_{y,r}\\
\dot \theta_y\\
\dot \omega_y\\
\end{array}
\right]
=
\left[
\begin{array}{c}
v_{x,r}\\
g \tan \theta_x - \hat a_x + d_{v_x}\\
-d_1 \theta_x + \omega_x\\
-d_0 \theta_x + n_0 a_x\\
v_{y,r}\\
g \tan \theta_y - \hat a_y + d_{v_y}\\
-d_1 \theta_y + \omega_y\\
-d_0 \theta_y + n_0 a_y\\
\end{array}
\right],
\end{array}\\
\end{aligned}
\end{equation}
where the states $(x_r,v_{x,r},y_r,v_{y,r})$ are the position and velocity of the 8D quadrotor in the frame of the double integrator. The remaining states $(\theta_x,\omega_x,\theta_y,\omega_y)$ are the pitch (rate) and roll (rate) of the 8D quadrotor.

\MCnote{Added offline reachability computation results and the range of velocity and positional TEBs.}   
\color{blue} notations for time-varying TEBs and Minkowski addition?modify Alg. 1\color{black}

\subsubsection{Online computation}
%
We utilize the MPC design introduced in \cite{Zhang2017} for the online path planning. See in Problem~\ref{pr: MPC}.
%
\begin{problem}\label{pr: MPC}
\begin{align*}
\min_{\mathbf{p},\mathbf{u}}  & \quad \sum^{N-1}_{k=0} l(p_k,u_k) + l_f(p_N-p_f)  \\
s.t. \quad & p_0 = p_{init},\\
&p_{k+1} = f_p(p_k,u_k),\\
& p_k \in \mathbb{P}_k ,\enspace u_k \in \mathbb{U},\\
& \mathbb{S}(p_k)\cap\constrAug(t_k) = \emptyset
\end{align*}
\end{problem}
where $l(\cdot,\cdot)$ and $l_f(\cdot)$ are convex stage and terminal cost functions, $N$ is the horizon for the MPC problem, and $t_k = t_0 + k \MCnote{\Delta t}$ denotes the current time used in simulation, with $t_0$ being the time when the MPC problem starts to be solved and $\Delta t$ the MPC sampling interval. The dynamical system $f_p(\cdot,\cdot)$ is set to be a discretized model of the 4D dynamics in \eqref{eq:Quad8D_dyn}. The state and input sequences along the horizon  are denoted by $\mathbf{p}=[p^{T}_0,p^{T}_1,\cdots,p^{T}_N]^{T}$ and $\mathbf{u}=[u^{T}_0,u^{T}_1,\cdots,u^{T}_{N-1}]^{T}$. The velocity states are subject to convex time-varying constraints:
%
\begin{equation}
(\hat v_{x,k},\hat v_{y,k}) \subset p_k \in \mathbb{P}_k :=\mathbb{P}\oplus\TEB_\pstate(t_k) \enspace ,
\end{equation}
%
where $\oplus$ denotes the Minkowski addition, $\mathbb{P}$ denotes the original state constraint, and $\TEB_\pstate(t_k)$ is the tracking error bound sampled at $t_k$, respectively. Given the state vector $p_k$, we denote the position of the controlled object by $(\hat x_k,\hat y_k) := \mathbb{S}(p_k)\subseteq \mathbb{R}^{2}$. To avoid collision with obstacles, $\mathbb{S}(p_k)$ is subject to the following constraint: 
%
\begin{equation}
\mathbb{S}(p_k)\cap\constrAug(t_k) = \emptyset \enspace ,
\end{equation}
%
with 
%
\begin{equation}
\constrAug(t_k) := \constrSense(t_0)\oplus\TEB_\pstate(t_k) \enspace .
\end{equation}
%
where $\constrSense(t_0)$ denotes the obstacles sensed at $t_0$.

In this paper, we represent the obstacles as polytopes, i.e., $\constrSense = \cap \constr^{i}$ with $\constr^{i}:= \{z\in\mathbb{R}^{n} \mid A^{i}z\leq b^{i}\}$ for $i = 1,\cdots ,M$. We follow the approach presented in \cite{Zhang2017} to compute a local minimal solution, by involving extra variables $\lambda^{i}$ for each obstacle $\constr^{i}$ and reformulating the collision avoidance constraint equivalently as follows: 
%
\begin{equation}
\exists \lambda^{i} >0, \; \mbox{s.t.} \; (A^{i} \mathbb{S}(p_k) - b^{i})^{T}\lambda^{i}  > 0, \; \|A^{i^{T}}\lambda^{i}\|_2\leq 1\enspace .
\end{equation}
%
Note that the collision avoidance constraint causes the MPC problem to be non-convex and thus computationally expensive.

The procedure of finding the next state of the planning system using the proposed MPC planner is summarized in Algorithm \ref{alg:mpc}.
%
\begin{algorithm}	
	\caption{MPC Path Planner Block}
	\label{alg:mpc}
	\begin{algorithmic}[1]
		\STATE \textbf{Initialization}:
 		\STATE Set initial time and states: $t_0 \leftarrow \tvar, \pstate_0 \leftarrow \pstate$
		\IF{MPC is ready to re-plan}
			\STATE Solve MPC for the optimal control sequence: $\mathbf{u}_t \leftarrow \text{Problem\ref{pr: MPC}} (t_0,p_0,\constrAug)$
		\ENDIF
        \STATE Get the current control: $u(t) \leftarrow u_k \in \mathbf{u}_t$ such that $t \in [t_0 + k \Delta t, t_0 + (k+1) \Delta t]$
		\STATE Output the next state: $\pstate_\text{next} = f_p(p,u(t))$
	\end{algorithmic}
\end{algorithm}

\subsubsection{Simulations}

The values for parameters $d_0,d_1,n_0,k_T,g$ used for the 8D model are: $d_0=10,d_1=8,n_0=10,k_T=0.91,g=9.81$.

The 8D control bounds are $|a_x|,|a_y|\leq10$ degrees.

The 4D control bound is $\|(\hat a_x,\hat a_y)\|_2\leq1.0$ m/s$^{2}$.

The disturbance bounds are $|d_{v_x}|,|d_{v_y}|\leq0.1$ m/s.

The sensing range is $\MCnote{r = 5}$ meters.

For the MPC problem we used a horizon $N=8$ with sampling interval $\Delta t = 0.2$ s.

Implementation of the MPC planner was based on MATLAB and \texttt{ACADO Toolkit} \cite{Houska2011a}. The nonlinear MPC problem was solved using an online active set strategy implemented in \texttt{qpOASES} \cite{Ferreau2014}. All the simulation results were obtained on a laptop with Ubuntu 14.04 LTS operating system and a Core i5-4210U CPU. The MPC planner re-plans every 0.8 s with an average computational time of 0.37 s for each planning loop. The frequency of control was once every 0.1 s for the 8D quadrotor system.

\MCnote{Simulation figures and explanations to be added.}
% Numerical Simulations (1-2p)
%% demonstrate feasibility (~.5)
%% real-time computation load (~.5)
%% comparison to other methods (~.5)

% !TEX root = tracking.tex
\section{Conclusions and Future work}
In this paper we have introduced the novel tool {FaSTrack}: Fast and Safe Tracking. This tool can be used to add robustness to various path and trajectory planners without sacrificing fast online computation. So far this tool can be applied to unknown environments with a limited sensing range and static obstacles. We are excited to explore several future directions for FaSTrack in the near future, including exploring robustness for moving obstacles, adaptable error bounds based on external disturbances and obstacle density, and demonstration on a variety of planners.

% Conclusion (0.5p)

%%%%%%%%%%%%%%%%%%%%%%%%%%%%%%%%%%%%%%%%%%%%%%%%%%%%%%%%%%%%%%%%%%%%%%%%%%%%%%%%
%\addtolength{\textheight}{1cm}   % This command serves to balance the column lengths
                                  % on the last page of the document manually. It shortens
                                  % the textheight of the last page by a suitable amount.
                                  % This command does not take effect until the next page
                                  % so it should come on the page before the last. Make
                                  % sure that you do not shorten the textheight too much.

\bibliographystyle{IEEEtran}
\bibliography{references}
\end{document}
