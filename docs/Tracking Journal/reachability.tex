% !TEX root = tracking.tex
\section{Robust-Invariant Region \label{sec:reachability}}
Goals overview: Compute invariant sets, given by the optimization solution

Properties of solution: 

\subsection{Trajectory Tracking as a Pursuit-Evasion Game}
Dynamics of system
\begin{equation}
\dot\vstate = \vdyn(\vstate, \vctrl, \dstb)
\end{equation}

Dynamics of trajectory
\begin{equation}
\dot\tstate = \tdyn(\tstate, \tctrl)
\end{equation}

Assume $\tstate$ is a subset of $\vstate$. Let $\tvind$ be the set of indices such that $\tstate_i = \vstate_{\tvind_i}$.

Relative state:
\begin{equation}
\rstate = \vstate - \tvmat\tstate
\end{equation}

Relative dynamics
\begin{equation}
\dot\rstate = \rdyn(\rstate, \vctrl, \tctrl, \dstb)
\end{equation}

Tracking error: $\rstate_{\tvind}$.

The goal of the system is to minimize the tracking error.

The goal of the trajectory, which is a ``virtual'' vehicle, is to maximize the tracking error.

\subsection{Optimization Problem}
Define an error function $\errfunc(\rstate)$

\begin{equation}
\valfunc(\rstate, t) = \max_{\vctrl(\cdot)} \min_{\tctrl(\cdot), \dstb(\cdot)} \min_{\tau \in [t, T]} \errfunc(\rtraj(\tau; \rstate, t, \vctrl(\cdot), \tctrl(\cdot), \dstb(\cdot))) 
\end{equation}

Take $T \rightarrow \infty$:

\begin{equation}
\valfunc_\infty(\rstate) = \max_{\vctrl(\cdot)} \min_{\tctrl(\cdot), \dstb(\cdot)} \min_{\tau>t} \errfunc(\rtraj(\tau; \rstate, t, \vctrl(\cdot), \tctrl(\cdot), \dstb(\cdot))) 
\end{equation}

\begin{thm}
Let $t_0 \geq 0$, and $\rstate_0$ be some state. Then

\begin{equation}
\begin{aligned}
&\max_{\vctrl(\cdot)} \min_{\tctrl(\cdot), \dstb(\cdot)} \min_{\tau \ge t_0} \valfunc_\infty(\rtraj(\tau; \rstate_0, t_0, \vctrl(\cdot), \tctrl(\cdot), \dstb(\cdot))) \\
&\qquad = \valfunc_\infty(\rstate_0)~ \forall t \ge t_0
\end{aligned}
\end{equation}
\end{thm}

\begin{proof}
It is clear that $\forall t > t_0$,
  
\begin{equation}
\begin{aligned}
\valfunc_\infty(\rstate(t)) &=\max_{\vctrl(\cdot)} \min_{\tctrl(\cdot), \dstb(\cdot)} \min_{\tau\ge t} \errfunc(\rtraj(\tau; \rstate(t), t, \vctrl(\cdot), \tctrl(\cdot), \dstb(\cdot))) \\
&\ge \max_{\vctrl(\cdot)} \min_{\tctrl(\cdot), \dstb(\cdot)} \min_{\tau\ge t_0} \errfunc(\rtraj(\tau; \rstate(t), t, \vctrl(\cdot), \tctrl(\cdot), \dstb(\cdot)))\\ 
&= \bar\valfunc_0
\end{aligned}
\end{equation}  

\noindent since the only the minimization over $\tau$ in $\bar\valfunc_1$ does not include the interval $[t_0, t)$.

\end{proof}

\begin{thm}

\begin{equation}
\max_{\vctrl(\cdot)} \min_{\tctrl(\cdot), \dstb(\cdot)} \min_{\tau>t} \errfunc(\rtraj(\tau; \rstate(t), t, \vctrl(\cdot), \tctrl(\cdot), \dstb(\cdot)))
\end{equation}


\end{thm}

\subsection{Dynamics of a Geometric Path}

\subsection{Solving the Optimization}

HJ Reachability (~1p)

Relative dynamics, setup, etc. (~1p)

Capture basin computation (~0.5p)