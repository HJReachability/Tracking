% !TEX root = tracking.tex
\section{Problem Formulation \label{sec:formulation}}
In this paper we seek to simultaneously plan and track a trajectory (or path converted to a trajectory) online and in real time. The planning is done using a kinematic or dynamic planning model. The tracking is done by a tracking model representing the autonomous system. The environment may contain static obstacles that are either known a priori or can be observed by the system within a limited sensing range (see Section \ref{sec:online}). In this section we will define the tracking and planning models, as well as the goals of the paper.

\subsection{Tracking Model}
The tracking model is a representation of the true autonomous system dynamics, and in general may be nonlinear and high-dimensional. Let $\tstate$ represent the state variables of the tracking model. The evolution of the dynamics satisfy the ordinary differential equation: 
\begin{equation}
\begin{aligned}
\label{eq:tdyn}
\frac{d\tstate}{d\tvar} = \dot{\tstate} = \tdyn(\tstate, \tctrl, \dstb), \tvar \in [0, \thor] \\
\tstate \in \tset, \tctrl \in \tcset, \dstb \in \dset
\end{aligned}
\end{equation}
We assume that the system dynamics $\tdyn : \tset\ \times\ \tcset \times \dset \rightarrow \tset$ are uniformly continuous, bounded, and Lipschitz continuous in $\tstate$ for fixed control $\tctrl$. The control function $\tctrl(\cdot)$ and disturbance function $\dstb(\cdot)$ are drawn from the following sets:
\begin{equation}
\begin{aligned}
\tctrl(\cdot) \in \tcfset(t) = \{\phi: [0, \thor] \rightarrow \tcset: \phi(\cdot) \text{ is measurable}\}\\
\dstb(\cdot) \in \dfset(t) = \{\phi: [0, \thor] \rightarrow \dset: \phi(\cdot) \text{ is measurable}\}
\end{aligned}
\end{equation}
where $\tcset, \dset$ are compact and $t\in[0, \thor]$ for some $T>0$. Under these assumptions there exists a unique trajectory solving (\ref{eq:tdyn}) for a given $\tctrl(\cdot) \in \tcset$ \cite{Coddington84}. The trajectories of (\ref{eq:tdyn}) that solve this ODE will be denoted as $\ttraj(\tvar; \tstate, \tvar_0, \tctrl(\cdot))$, where $\tvar_0,\tvar \in [0, \thor]$ and $\tvar_0 \leq \tvar$. These trajectories will satisfy the initial condition and the ODE (\ref{eq:tdyn}) almost everywhere:
\begin{equation}
\label{eq:fdyn_traj}
\begin{aligned}
\frac{d}{d\tvar}\ttraj(\tvar; \tstate, \tvar_0, \tctrl(\cdot)) &= \tdyn(\ttraj(\tvar; \tstate, \tvar_0, \tctrl(\cdot)), \tctrl(\tvar)) \\
\ttraj(\tvar; \tstate, \tvar, \tctrl(\cdot)) &= \tstate
\end{aligned}
\end{equation}

\subsection{Planning Model}
The planning model is used by the path or trajectory planner to solve for the desired path online. The dynamics of the system can be kinematic or dynamic depending on the requirements of the planner. Let $\pstate$ represent the state variables of the planning model, with control $\pctrl$. The planning states $\pstate \in \pset$ are a subset of the tracking states $\tstate \in \tset$. The dynamics similarly satisfy the ordinary differential equation:
\begin{equation}
\begin{aligned}
\label{eq:pdyn}
\frac{d\pstate}{d\tvar} = \dot{\pstate} = \pdyn(\pstate, \pctrl), \tvar \in [0, \thor], \pstate \in \pset, \ \underline{\pctrl} \leq \pctrl \leq \bar{\pctrl}
\end{aligned}
\end{equation}
Note that the planning model does not involve a disturbance input. This is a key feature of FASTrackHD: the treatment of disturbances is only necessary in the tracking model, which is modular with respect to any planning method, including those that do not account for disturbances.

\subsection{Goals of This Paper}
The goals of the paper are threefold:
\begin{enumerate}
	\item To provide a tool for precomputing functions (or look-up tables) to determine a guaranteed tracking error bound between tracking and planning models, and optimal safety controller for robust motion planning with nonlinear dynamic systems
	\item To develop a framework for easily implementing this tool with fast real-time path and trajectory planners.
	\item To demonstrate the tool and framework in an example using a high dimensional system
\end{enumerate}
