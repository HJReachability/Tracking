% !TEX root = tracking.tex
\section{Problem Formulation \label{sec:formulation}}
In this paper we seek to plan a path or trajectory online and in real-time using a general, potentially kinematic planning system. This planner is then followed with a tracking system. The environment may contain static obstacles that are known or unknown with a limited sensing range on the system. In this section we will define the planning model, tracking model, and environment. We will conclude the section with the goals of the paper.

\subsection{Planning Model}


\subsection{Tracking Model}
Let $\tstate$ be the state variable the real system. The evolution of the dynamics satisfies the ordinary differential equation:

\begin{equation}
\begin{aligned}
\label{eq:fdyn}
\frac{d\tstate}{d\tau} = \dot{\tstate} = \tdyn(\tstate, \tctrl, \dstb), \tau \in [t, 0] \\
\tstate \in \tset, \tctrl \in \tcset, \dstb \in \dset
\end{aligned}
\end{equation}

We assume that the system dynamics $\tdyn : \tset\ \times\ \tcset \rightarrow \tset$ are uniformly continuous, bounded, and Lipschitz continuous in $\tstate$ for fixed control $\tctrl$. The control function $\tctrl(\cdot)$ is drawn from the set of measurable functions\footnote{A function $f:X\to Y$ between two measurable spaces $(X,\Sigma_X)$ and $(Y,\Sigma_Y)$ is said to be measurable if the preimage of a measurable set in $Y$ is a measurable set in $X$, that is: $\forall V\in\Sigma_Y, f^{-1}(V)\in\Sigma_X$, with $\Sigma_X,\Sigma_Y$ $\sigma$-algebras on $X$,$Y$.}:
\begin{equation}
\begin{aligned}
\tctrl(\cdot) \in \tcfset(t) = \{\phi: [t, 0] \rightarrow \tcset: \phi(\cdot) \text{ is measurable}\}
\end{aligned}
\end{equation}

Under these assumptions there exists a unique trajectory solving \ref{eq:fdyn} for a given $\tctrl(\cdot) \in \tcset$ \cite{Coddington84}. The trajectories of \ref{eq:fdyn} that solve this ODE will be denoted as $\tctrl(\cdot)$ as $\ttraj(\tau; \tstate, t, \tctrl(\cdot))$. These trajectories will satisfy the initial condition and the ODE \ref{eq:fdyn} almost everywhere:
\begin{equation}
\label{eq:fdyn_traj}
\begin{aligned}
\frac{d}{d\tau}\ttraj(\tau; \tstate, t, \tctrl(\cdot)) &= \tdyn(\ttraj(\tau; \tstate, t, \tctrl(\cdot)), \tctrl(\tau)) \\
\ttraj(t; \tstate, t, \tctrl(\cdot)) &= \tstate
\end{aligned}
\end{equation}

\SHnote{explain what trajectory notation means better}

\subsection{Planning Environment}
\textcolor{red}{Requires static obstacles that aren't too densely clustered. Depending on path planner, environment could unknown with a sensing radius, or just a known environment}

\subsection{Goals of This Paper}
\textcolor{red}{
	a) Provide look-up table(s) where input is relative state of your true system to your path planner, and the outputs are tracking error bound and optimal safety control for your real system.\\
	b) Develop a framework for using the look-up table(s) that is simple and easy to use for a variety of path planners\\
	c) Demonstrate using a 10D model tracking an RRT path planner}