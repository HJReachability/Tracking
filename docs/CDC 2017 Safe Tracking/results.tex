% !TEX root = tracking.tex
\section{10D Quadrotor RRT Example \label{sec:results}}
\begin{figure*}
	\centering
	\includegraphics[width=0.9\textwidth]{fig/quad10D_example_cost}
	\caption{\textcolor{red}{2D projections of reward and value functions, along with corresponding 3D positional projections of initial state and tracking error bound}}
	\label{fig:quad10D_example}
\end{figure*} 

\begin{figure}
	\centering
	\includegraphics[width=0.4\textwidth]{fig/quad10D_slices}
	\caption{\textcolor{red}{various 3D slices}}
	\label{fig:quad10D_example}
\end{figure} 
We demonstrate this framework with a 10D near-hover quadrotor developed in \cite{Bouffard12} tracking a 3D point source path generated by a rapidly-exploring random tree (RRT). First we precompute the look-up table(s), then we set up the RRT to convert paths to simple 3D trajecotories. Finally we implement the online framework to navigate the 10D system through a 3D environment with static obstacles.

\subsection{Precomputation of 10D-3D system}
First we define the 10D dynamics of the real quadrotor and the 3D dynamics of the simple point source:

\begin{equation}
\label{eq:Quad10D_dyn}
\begin{aligned}
\begin{array}{c}
\left[
\begin{array}{c}
\dot{x}\\
\dot{v_x}\\
\dot{\theta_x}\\
\dot\omega_x\\
\dot{y}\\
\dot{v_y}\\
\dot{\theta_y}\\
\dot\omega_y\\
\dot{z}\\
\dot{v_z}
\end{array}
\right]
=
\left[
\begin{array}{c}
v_x\\
g \tan \theta_x\\
-d_1 \theta_x + \omega_x\\
-d_0 \theta_x + n_0 a_x\\
v_y\\
g \tan \theta_y\\
-d_1 \theta_y + \omega_y\\
-d_0 \theta_y + n_0 a_y\\
v_z \\
k_T a_z - g
\end{array}
\right]
\left[
\begin{array}{c}
\dot{x}\\
\dot{y}\\
\dot{z}\\
\end{array}
\right]
=
\left[
\begin{array}{c}
b_x\\
b_y\\
b_z \\
\end{array}
\right]
\end{array}\\
\end{aligned}
\end{equation}

where states $(x, y, z)$ denote the position, $(v_x, v_y, v_z)$ denote the velocity, $(\theta_x, \theta_y)$ denote the pitch and roll, and $(\omega_x, \omega_y)$ denote the pitch and roll rates. The controls of the 10D system are $(a_x, a_y, a_z)$, where $a_x$ and $a_y$ represent the desired pitch and roll angle, and $a_z$ represents the vertical thrust. The 3D system controls are $(b_x, b_y, b_z)$, and represent the velocity in each positional dimension. Next the relative dynamics between the two systems is defined, with the addition of external disturbances:

\begin{equation}
\label{eq:Quad10DRel_dyn}
\begin{aligned}
\begin{array}{c}
\left[
\begin{array}{c}
\dot{x_r}\\
\dot{v_{xr}}\\
\dot{\theta_{xr}}\\
\dot\omega_{xr}\\
\dot{y_r}\\
\dot{v_{yr}}\\
\dot{\theta_{yr}}\\
\dot\omega_{yr}\\
\dot{z_r}\\
\dot{v_{zr}}
\end{array}
\right]
=
\left[
\begin{array}{c}
v_x - b_x + d_x\\
g \tan \theta_x\\
-d_1 \theta_x + \omega_x\\
-d_0 \theta_x + n_0 a_x\\
v_y - b_y + d_y\\
g \tan \theta_y\\
-d_1 \theta_y + \omega_y\\
-d_0 \theta_y + n_0 a_y\\
v_z - b_z + d_z\\
k_T a_z - g
\end{array}
\right]
\end{array}\\
\end{aligned}
\end{equation}

In the above system the external disturbance $(d_x, d_y, d_z)$ is caused by wind, which acts on the velocity in each dimension. The values for parameters $d_0,d_1,n_0,k_T,g$ used were: $d_0=10,d_1=8,n_0=10,k_T=0.91,g=9.81$. The 10D control bounds were $|a_x|,|a_y|\leq10$ degrees, $0\leq a_z\leq 1.5g$ m/s$^{2}$. The 3D control bounds were $|b_x|,|b_y|,|b_z|\leq0.5$ m/s. The disturbance bounds were $|d_x|,|d_y|,|d_z|\leq0.1$ m/s.

Next we follow the setup describe in section \ref{sec:precomp} to create a reward function, which we then evaluate using HJ reachability until convergence to produce the invariant value function. Historically this 10D nonlinear relative system would be intractable for HJ reachability analysis, but using new methods \textcolor{red}{cite decomposition}, we can decompose this system into 3 subsystems (one for each positional dimension). To do this we must also decompose the reward function, resulting in a one-norm instead of a two-norm. This reward function as well as the resulting value function can be seen projected onto the $x,y$ dimensions in Figure \ref{fig:quad10D_example}.

Figure \ref{fig:quad10D_example} also shows 3D positional projections of the mapping between initial relative state to maximum potential relative distance over all time (i.e. tracking error bound). If the real system starts exactly at the origin in relative coordinates, its tracking error bound will be a box of 0.90 meters in each direction. We save the look-up tables of the value function (to define tracking bound) and its spatial gradients (to compute optimal control of the real system).

\subsection{RRT Online Planning}
\textcolor{red}{what RRT planner we're using, how we convert it to dynamics, setup of environment, results}
