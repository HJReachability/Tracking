% !TEX root = tracking.tex
\section{Related Work \label{sec:relatedwork}}
Motion planning is a very active area of research in the controls and robotics communities.  In this section we will discuss past work on path planning, kinematic planning, and dynamic planning.  A major current challenge is to find an intersection of robust, dynamic, and real-time planning. 

There is a wealth of research in the area of sample-based path planning.  Planning methods like Rapidly-Exploring Random Trees (RRT) [Kuffner2000], Probabilistic Road Maps (PRM) [Kavraki1996], and Fast Marching Tree (FMT) [Janson2015] can find collision-free paths through known or partially known environments. These paths can then be smoothed with shortcut-based methods or turned into optimal motion plans [Richter2016, Karaman 2011, Kobilarov 2012].  These systems work well in many applications, but are not designed to be robust to model uncertainty or disturbances.

Motion planning for kinematic systems can also be accomplished through online trajectory optimization using methods such as TrajOpt [Schulman2013] and CHOMP [Ratliff2009].  \SHnote{These methods use simple dynamics to converge quickly to solutions using convex sequential quadratic programming.}

Model Predictive Control (MPC) has been a very successful method for dynamic trajectory optimization in both academia and industry [Qin2003].  However, combining speed, safety, and complex dynamics is a difficult balance to achieve.  Using MPC for robotic and aircraft systems typically requires reducing the system complexity or restricting the system to evolve more slowly over time [Bellingham2002, Diehl2002, Grune2011].  Robustness in linear systems can be achieved using constraint tightening MPC to balance speed with safety [Kuwata2007, Richards2006]. Tube MPCs perform robust nonlinear planning by applying a nonlinear trajectory tracking system [Mayne2011]; though promising, this method has not yet been applied to real-time planning. \SHnote{check out borelli's book, find latest references}

There are other methods of dynamic trajectory planning that manage to cleverly skirt the issue of solving for optimal trajectories online.  One such method works by storing a fixed precomputed set of trajectories called motion primitives that are then selected and composed together online.  This has been remarkably useful in many practical applications [Gillula2010, Dey2014, Barry2016], and there is impressive work on making these systems robust using funnels [Majumdar2016].  Another tactic for online dynamic trajectory planning involves the generation of several random trajectories at each waypoint, and then picking the best of those computed [Kalakrishnan2011, Schwesinger2013, Krusi2015].  This method works well in many applications, but is risky in its reliance on finding a randomly-generated collision-free trajectory.  \SHnote{Also mention control barrier functions [Xu2015, Ames2014], and contraction theory stuff [Singh2017].}

Offline planners like Hamilton-Jacobi reachability analysis can find control policies and guarantees for nonlinear systems that avoid obstacles and are robust to bounded disturbances [Mitchell2005].  However, this method can only approach real-time speed for very low-dimensional (1D-2D) systems. Although there has been work to speed up analysis by decomposing high-dimensional systems into smaller subsystems [Chen2016, Chen2017], dimensionality is still a common hurdle.

The work presented in this paper differs from the robust planning methods above because our model is designed to be modular and easy to use in conjunction with a number of path and trajectory planners.  Our model allows planners to use simplified dynamics to attain real-time planning while simultaneously maintaining safety and providing the optimal control for the nonlinear system to follow the planner. We can handle bounded external disturbances (e.g. wind) and work with both known and unknown environments (within a predetermined sensing range, described in section \SHnote{SECTION}) with static obstacles. 

