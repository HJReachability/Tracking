%%%%%%%%%%%%%%%%%%%%%%%%%%%%%%%%%%%%%%%%%%%%%%%%%%%%%%%%%%%%%%%%%%%%%%%%%%%%%%%%
%2345678901234567890123456789012345678901234567890123456789012345678901234567890
%        1         2         3         4         5         6         7         8

%\documentclass[letterpaper, 10 pt, conference]{ieeeconf}  % Comment this line out
                                                          % if you need a4paper
\documentclass[letterpaper, 10pt, conference]{ieeeconf}      % Use this line for a4
                                                          % paper

 
\IEEEoverridecommandlockouts                              % This command is only
                                                          % needed if you want to
                                                          % use the \thanks command
\overrideIEEEmargins
% See the \addtolength command later in the file to balance the column lengths
% on the last page of the document

\usepackage{amsmath}    % need for sub equations
\usepackage{amsfonts}
\usepackage{graphicx}   % need for figures
\usepackage{subcaption}
\usepackage{epsfig} 
\usepackage{color}
\usepackage[normalem]{ulem}
\usepackage{cancel}
\usepackage{amssymb}
\usepackage{color}
%\usepackage{my_macros}
\usepackage[ruled,vlined,titlenotnumbered]{algorithm2e} 

\newcommand{\R}{\mathbb{R}}
\newcommand{\xset}{\mathcal{X}}
\newcommand{\yset}{\mathcal{Y}}
\newcommand{\xfset}{\mathbb{X}}
\newcommand{\yfset}{\mathbb{Y}}
\newcommand{\cset}{\mathcal{U}}
\newcommand{\cfset}{\mathbb{U}}
\newcommand{\dset}{\mathcal{D}}
\newcommand{\dfset}{\mathbb{D}}
\newcommand{\reachset}{\mathcal{V}}
\newcommand{\targetset}{\mathcal{L}}
\newcommand{\traj}{\zeta} % trajectory
\newcommand{\state}{z}
\newcommand{\zset}{\mathcal{Z}}
\newcommand{\ctrl}{u}

\newcommand{\tvar}{t}
\newcommand{\thor}{T} % Time horizon

\newcommand{\vstate}{x} % Vehicle state
\newcommand{\tstate}{y} % Trajectory state
\newcommand{\rstate}{z} % Relative state

\newcommand{\vtraj}{\xi_\vdyn}
\newcommand{\ttraj}{\xi_\tdyn}
\newcommand{\rtraj}{\xi_\rdyn}

\newcommand{\vctrl}{u} % Vehicle control
\newcommand{\dstb}{d} % Disturbance
\newcommand{\tctrl}{w} % Trajectory control

\newcommand{\vdyn}{f} % Vehicle dynamics
\newcommand{\tdyn}{g} % Trajectory dynamics
\newcommand{\rdyn}{r} % Relative dynamics

\newcommand{\tvind}{s} % Index of vehicle state corresponding to trajectory state
\newcommand{\tvmat}{R} % Matrix for transforming trajectory state to the same length as vehicle state

\newcommand{\errfunc}{l} % Error function
\newcommand{\valfunc}{V} % Value function

\newtheorem{thm}{Theorem}
\newtheorem{prop}{Proposition}

\newcommand{\MCnote}{\textcolor{red}}

\title{\LARGE \bf \MCnote{A Very Good Title}}

\author{authors
\thanks{\color{red}This work has been supported in part by NSF under CPS:ActionWebs (CNS-931843), by ONR under the HUNT (N0014-08-0696) and SMARTS (N00014-09-1-1051) MURIs and by grant N00014-12-1-0609, by AFOSR under the CHASE MURI (FA9550-10-1-0567). The research of M. Chen has received funding from the ``NSERC PGS-D'' Program. The research of S. Herbert has received funding from the NSF GRFP and the UC Berkeley Chancellor's Fellowship Program}
}

\begin{document}
\maketitle
\thispagestyle{empty}
\pagestyle{empty}

%%%
\begin{abstract}

Quadrotors have become very popular in research and industry for tasks that require exploration of unknown environments. However, path planning for autonomous vehicles is a computationally intensive task, and balancing speed of planning with safety and dynamic feasibility is challenging. Simplified models of quadrotor dynamics are easy to plan, but do not capture nonlinear behavior. Guaranteed safe paths can be computed for more realistic and complicated dynamics of quadrotors, but these paths require heavy computational load. We propose a method that combines these two approaches. We formulate a capture-avoid game between a simplified model used for path planning and the true vehicle dynamics. We then use reachability analysis to precompute a mapping from the current relative state between the two systems to the worst possible tracking error bound over time, providing a "safety bubble" around the simplified system. This mapping also uses the relative state to determine the optimal control for the real vehicle to reduce the tracking error. Once the precomputation is complete, we perform path planning in real-time using the simplified model augmented by this safety bubble that captures all possible deviations due to nonlinearities or external disturbances. We implement this using a 10D quadrotor model tracking a 3D linear RRT path planner. \color{red} results, etc.
\end{abstract}

% !TEX root = tracking.tex
\section{Introduction}

Autonomous systems have a great potential to improve many industries in the near future. However, to achieve their potential there is a need to ensure the ability to make real-time plans while maintaining safety guarantees.
% As unmanned aerial vehicles (UAVs) and other autonomous systems become more commonplace, it is essential that they be able to plan safe motion paths through crowded environments in real-time. 
This is particularly crucial for navigating through environments that are \textit{a priori} unknown, because replanning based on updated information about the environment is often necessary. 
Achieving safe navigation in real time is difficult for many common dynamical systems due to the computational complexity of generating and formally verifying the safety of dynamically feasible trajectories.
 In order to achieve real-time planning, many algorithms use highly simplified model dynamics or kinematics to create a nominal trajectory that is then tracked by the system using a feedback controller such as a linear quadratic regulator (LQR).  These nominal trajectories may not be dynamically feasible for the true autonomous system, resulting in a tracking error between the planned path and the executed trajectory.
 This concept is illustrated in Fig. \ref{fig:chasing}, where the path was planned using a simplified planning model, but the real dynamical system cannot track this path exactly. 
Additionally, external disturbances (e.g. wind) can be difficult to account for using real-time path or trajectory planning algorithms, causing another source of tracking error. 
These tracking errors can lead to dangerous situations in which the planned path is safe, but the actual system trajectory enters unsafe regions.  Therefore, real-time planning is achieved at the cost of guaranteeing safety.  Common practice techniques augment obstacles by an ad hoc safety margin (see Fig. \ref{fig:chasing}, right), which may alleviate the problem but is performed heuristically and therefore does not guarantee safety.
 \begin{figure}
 	\centering
 	\includegraphics[width=0.35\textwidth]{fig/chasing}
 	\caption{Left: A planning algorithm uses a fast but simple model (blue disk), to plan around obstacles (gray disks). The more complicated tracking model (green plane) tracks the path. By using FaSTrack the autonomous system is guaranteed to stay within some TEB (black circle). Right: Safety can be guaranteed by planning with respect to obstacles augmented by the TEB (large black circles).}
 	\label{fig:chasing}
 \end{figure}
 %Real-time planning that is both safe and accurate presents a very difficult challenge: accuracy and robustness in many dyanimcal systme sis difficult to compute, often precluding real-time computer hands.fast planning is generally at odds with the need for maintaining safety and robustness.  

To attain fast planning speed while maintaining safety, we propose the modular framework FaSTrack: Fast and Safe Tracking.  As before, FaSTrack allows path or trajectory planning algorithms to use a simplified model of the system in order to operate in real time using augmented obstacles.  However, the bound for augmenting obstacles is rigorously computed and comes with a corresponding optimal tracking controller. Together this bound and controller guarantees safety for the autonomous system as it tracks the simplified plans.

We compute this bound and controller by modeling the navigation task as a pursuit-evasion game between a sophisticated \textit{tracking model} (pusuer) and the simplified \textit{planning model} of the system (evader). 
The tracking model accounts for complex system dynamics as well as bounded external disturbances, while the simple planning model enables the use of real-time planning algorithms. 
Offline, the pursuit-evasion game between the two models can be analyzed using any suitable method. 
This results in a \textit{tracking error function} that maps the initial relative state between the two models to the \textit{tracking error bound} (TEB): the maximum possible relative distance that could occur over time. 
This TEB can be thought of as a ``safety bubble" around the planning model of the system that the tracking model of the system is guaranteed to stay within.

When this precomputation converges, an invariant TEB can be computed for all time.  Since the planning model can be designed by the user, typically one can select a model such that the computation converges.  However, there may be cases in which convergence doesn't occur (i.e. even when acting optimally the autonomous sytem cannot keep up with the planning model used by the path or trajectory planning algorithm).  In these cases we can instead compute a time-varying TEB.  Intuitively, this means that as time progresses the tracking error bound increases by a known amount.

Because the tracking error is bounded in the relative state space, we can precompute and store the \textit{optimal tracking controller} that  maps the real-time relative state to the optimal tracking control for the tracking model of the sytstem to pursue the planning model of the system. 
The offline computations are \textit{independent} of the path planned in real time.

Online, the autonomous system senses local obstacles, which are then augmented by the TEB to ensure that no potentially unsafe paths can be computed. 
Next, any chosen path or trajectory planning algorithm uses the simplified planning model and the local environment to determine the next desired state. 
The autonomous system (represented by the tracking model) then finds the relative state between itself and the next desired state. 
If this relative state is nearing the TEB then it is plugged into the optimal tracking controller to find the instantaneous optimal tracking control of the tracking model required to stay within the error bound; otherwise, any tracking controller may be used. In this sense, FaSTrack provides a \emph{least-restrictive} control law.
This process is repeated as long as desired. 

FaSTrack was designed to be modular, and can be used with any method for computing the TEB in conjunction with any existing fast path or trajectory planning algorithms.  
This enables motion planning that is real-time, guaranteed safe, and dynamically accurate. 
In this paper, we demonstrate the FaSTrack framework by using three different real-time planning algorithms that have been ``robustified" by precomputing the TEB and tracking controller. 
The planning algorithms used in our numerical examples are the fast sweeping method (FSM) \cite{Takei2013}, rapidly-exploring random trees (RRT) \cite{Kuffner2000,Kavraki1996}, and model-predictive control (MPC) \cite{Qin2003}. 
In the three examples, we also consider different tracking and planning models.
The precomputation of the TEB and optimal tracking control function for each planning-tracking model pair is done by solving a Hamilton-Jacobi (HJ) partial differential equation (PDE). 
Two of the precomputations converge to an invariant TEB, and one uses a time-varying TEB.
In the simulations, the system travels through a static environment with constraints defined, for example, by obstacles, while experiencing disturbances.
The constraints are only fully known through online sensing, for example, once obstacles are within the limited sensing region of the autonomous system. 
By combining the TEB with real-time planning algorithms, the system is able to safely plan and track a trajectory through the environment in real time. 
% Introduction (.5-1p)
%%Tracking with quadrotors is a need
%%There exist methods that work in real time and methods that work for safety but not very many for both
%%Goal: combine both in a simple way

% !TEX root = tracking.tex
\section{Related Work \label{sec:relatedwork}}
\textcolor{red}{work on fast planning\\
work on safe planning\\
work on both\\
how ours is different}
% Related Work (1p)
%%work on fast planning
%%work on safe planning
%%work on both
%%how ours is different

% !TEX root = tracking.tex
\section{General Framework \label{sec:framework}}
Given a dynamical system, we propose a hierarchical framework for combining Hamilton-Jacobi safety analysis with planning methods in a modular way.

Given: Target, mechanism for sensing obstacles

Goal: reach target without colliding with obstacles

Offline: compute bubble and error-feedback controller (tracker)

Online: At every time iteration,
\begin{enumerate}
  \item Sense and update obstacles (can also be done every N iterations)
  \item Augment obstacles according to bubble
  \item Plan path or trajectory using planner, assuming currently sensed and augmented obstacles
  \item Robustly track trajectory using tracker
\end{enumerate}


\textbf{Maybe put next paragraph in the introduction}

There are many fast planners that could potentially do planning in real-time; however, these typically cannot account for disturbances in a provably safe way. In addition, complex system models with nonlinear dynamics complicate planning algorithms (non-convex for MPC, more difficult for RRT). On the other hand, HJ reachability is able to handle disturbances, and is agnostic to system dynamics. In addition, provably guarantees can be provided. However, HJ reachability and in general formal verification methods can be very expensive to compute.

Refer to figure: planning level and safety level. 

In the safety level, we start with the error dynamics, and we compute two things: bubble which is fed into planner to plan with extra margin, and error-feedback controller for real-time control. These two can be computed offline independent of the planned path.

In the planning level, any planning method such as MPC, RRT, etc. (cite some things) can be used. The planning level does not need to take into account disturbances, and can use simple system dynamics or even no dynamics at all. In fact we will be using a simple RRT planner which simply provides paths, in the form of a sequence of line segments, which are not dynamically feasible. 

\begin{figure}
\includegraphics[width=\columnwidth]{fig/framework_online}
\caption{Online framework}
\label{fig:fw_online}
\end{figure}

\begin{figure}
  \includegraphics[width=\columnwidth]{fig/framework_offline}
  \caption{Offline framework}
  \label{fig:fw_offline}
\end{figure}

\begin{figure}
  \includegraphics[width=\columnwidth]{fig/hybrid_controller}
  \caption{Hybrid controller}
  \label{fig:hybrid_ctrl}
\end{figure}

% !TEX root = tracking.tex
\section{Computing Tracking Safety Radius \label{sec:reachability}}
To precompute the tracking bound we must set up a capture-avoid game between the real and virtual vehicles, which we then analyze using HJ reachability. In this game, the real system will try to "capture" the virtual system, while the virtual system is doing everything it can to avoid capture. By using reachability to analyze this game we will get a guaranteed bound on how far apart the two vehicles will ever be even when the virtual system is acting as inconveniently as possible.

\subsection{Individual and Relative Dynamics}

Let $z_1$ be the state variable of the virtual system used for MPC planning, and $z_2$ be the state variable of the real system. The evolution of these states satisfies their respective ordinary differential equations:

\begin{equation}
\begin{aligned}
\label{eq:fdyn}
\frac{d\state_i}{ds} = \dot{\state_i} = f_i(\state_i, \ctrl_i), s \in [t, 0] \\
\state_i \in \zset_i, \ctrl_i \in \cset_i, i = 1,2
\end{aligned}
\end{equation}

We assume that the system dynamics $f_i : \zset_i\ \times\ \cset_i \rightarrow \zset_i$ are uniformly continuous, bounded, and Lipschitz continuous in $z_i$ for fixed control $u_i$. The control functions functions $u_i(\cdot)$ are drawn from the set of measurable functions\footnote{A function $f:X\to Y$ between two measurable spaces $(X,\Sigma_X)$ and $(Y,\Sigma_Y)$ is said to be measurable if the preimage of a measurable set in $Y$ is a measurable set in $X$, that is: $\forall V\in\Sigma_Y, f^{-1}(V)\in\Sigma_X$, with $\Sigma_X,\Sigma_Y$ $\sigma$-algebras on $X$,$Y$.}:
\begin{equation}
\begin{aligned}
\ctrl_i(\cdot) \in \cfset_i(t) = \{\phi: [t, 0] \rightarrow \cset_i: \phi(\cdot) \text{ is measurable}\}\\
i = 1,2
\end{aligned}
\end{equation}

Under these assumptions there exists a unique trajectory solving \ref{eq:fdyn} for a given $u_i(\cdot) \in \cset_i$ \cite{Evans84}. The trajectories of \ref{eq:fdyn} that solve this ODE will be denoted as $\ctrl(\cdot)$ as $\traj_i(s; \state_i, t, \ctrl_i(\cdot)), i = 1,2$. These trajectories will satisfy the initial condition and the ODE \ref{eq:fdyn} almost everywhere:
\begin{equation}
\label{eq:fdyn_traj}
\begin{aligned}
\frac{d}{ds}\traj_i(s; \state_i, t, \ctrl_i(\cdot)) &= f_i(\traj_i(s; \state_i, t, \ctrl_i(\cdot)), \ctrl_i(s)) \\
\traj_i(t; \state_i, t, \ctrl_i(\cdot)) &= \state_i, \ i = 1,2
\end{aligned}
\end{equation}

We now have the dynamics for the individual systems, but to set up the capture-avoid game we must first define the relative states and dynamics. We place the virtual vehicle at the origin by subtracting its states $(z_1)$ from the real system's states $(z_2)$. In this frame of reference $(z_r)$ we are given the states of the real system relative to the virtual system.

\begin{equation}
\begin{aligned}
z_r =& z_2 - z_1 \\
g(z_r,u_1,u_2) =& f_2(z_2,u_2) - f_1(z_1,u_1)
\end{aligned}
\end{equation}

The relative dynamics will include the relative position states $x_r, y_r$ and any other relevant states such as relative angles and velocities.

\subsection{Formalizing the Capture-Avoid Game}
Now that we have the relative dynamics between the two systems we must define a metric for the tracking error bound between these systems. We do this by defining an implicit surface function as a cost function in the new frame of reference. Because the metric we care about is distance to the origin, this cost function is a simple signed distance function in position space centered at the origin:
\begin{equation}
l(x_r,y_r)= \| [x_r,y_r] \|_2
\end{equation}

\begin{figure}
	\centering
	\includegraphics[width=0.47\textwidth]{fig/cost_function}
	\caption{filler image about the implicit surface function}
	\label{fig:cost}
\end{figure} 

 This can be seen in Figure \ref{fig:cost}, where the rings represent varying level sets of the cost function. The real vehicle will try to minimize this cost to reduce the relative distance, while the virtual vehicle will do the opposite.
 
 We want to find the farthest distance (and thus highest cost) that this game will ever reach when both players are acting optimally. Therefore we want to find a mapping between the initial state of the system and the maximum cost achieved over the time horizon. This mapping is through our value function, defined as:
 \begin{equation}
 	V(\state_r)= \min_{u_2} \max_{u_1} \max_{t\in [0,T]} l(x_r(t),y_r(t))
 \end{equation} 
 
 This is a modified version of the Hamilton-Jacobi formulation as described by \textcolor{red}{cite jaime,mo, time varying reachability}. \textcolor{red}{explain why this is equivalent to taking the min (or in the way I wrote it, max) between the current value function and the target set}
 
 run to desired time or convergence
 conditions for convergence
 
 results in Figure \ref{fig:value}
 
 
 \begin{figure}
 	\centering
 	\includegraphics[width=0.47\textwidth]{fig/value_function}
 	\caption{filler image about the value function}
 	\label{fig:value}
 \end{figure} 
 guarantee to remain within current level set (proof)
 if p1 does not act optimally, can get into a closer bound
 eventual limit
 
 can also add external disturbances easily
 
% Computing capture basin (~2.5p)
%% HJ Reachability (~1p)
%% Relative dynamics, setup, etc. (~1p)
%% Capture basin computation (~0.5p)

% !TEX root = tracking.tex
\section{Fast Path Planning using Rapidly-Exploring Random Tree \label{sec:rrt}}
Potential methods to use (~.5p)

Dealing with obstacles (~.5p)
% Fast Path Planning using MPC (~1p)
%% Potential methods to use (~.5p)
%% Dealing with obstacles (~.5p)

% !TEX root = tracking.tex
\section{10D Quadrotor RRT Example \label{sec:results}}
\textcolor{red}{explain 10D computation, RRT planning and conversion to dynamics, putting the two together, results}
\subsection{10D}
\textcolor{red}{use decomposition to run this in 10D. Follows 3D super-simple dynamics. Parameters, results}
\begin{figure*}
	\centering
	\includegraphics[width=0.8\textwidth]{fig/quad10D_example2}
	\caption{\textcolor{red}{2D projections of reward and value functions, along with corresponding 3D positional projections of initial state and tracking error bound}}
	\label{fig:quad10D_example}
\end{figure*} 
\subsection{RRT Online Planning}
\textcolor{red}{what RRT planner we're using, how we convert it to dynamics, setup of environment, results}
% Numerical Simulations (1-2p)
%% demonstrate feasibility (~.5)
%% real-time computation load (~.5)
%% comparison to other methods (~.5)

% !TEX root = tracking.tex
\section{Conclusions and Future work}
In this paper we have introduced the novel tool {FaSTrack}: Fast and Safe Tracking. This tool can be used to add robustness to various path and trajectory planners without sacrificing fast online computation. So far this tool can be applied to unknown environments with a limited sensing range and static obstacles. We are excited to explore several future directions for FaSTrack in the near future, including exploring robustness for moving obstacles, adaptable error bounds based on external disturbances and obstacle density, and demonstration on a variety of planners.

% Conclusion (0.5p)

%%%%%%%%%%%%%%%%%%%%%%%%%%%%%%%%%%%%%%%%%%%%%%%%%%%%%%%%%%%%%%%%%%%%%%%%%%%%%%%%
%\addtolength{\textheight}{1cm}   % This command serves to balance the column lengths
                                  % on the last page of the document manually. It shortens
                                  % the textheight of the last page by a suitable amount.
                                  % This command does not take effect until the next page
                                  % so it should come on the page before the last. Make
                                  % sure that you do not shorten the textheight too much.

\bibliographystyle{IEEEtran}
\bibliography{references}
\end{document}
