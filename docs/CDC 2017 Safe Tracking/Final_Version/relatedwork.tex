% !TEX root = tracking.tex
\section{Related Work \label{sec:relatedwork}}
A major current challenge is to find an intersection of robust and real-time planning for general nonlinear systems. Sample-based planning methods like RRT \cite{Kuffner2000} and others \cite{Kavraki1996,Janson2015,Richter2016, Karaman2011, Kobilarov2012} can find collision-free paths through known or partially known environments. Although extremely effective in a number of use cases, these algorithms are not designed to be robust to model uncertainty or disturbances.

Motion planning for kinematic systems can also be accomplished through online trajectory optimization \cite{Schulman2013,Ratliff2009}. For dynamic systems, model predictive control (MPC) has been successful \cite{Qin2003}. However, combining speed, safety, and complex dynamics is a difficult balance to achieve; often model reduction and linearization are used to apply MPC effectively \cite{Vitus2008, Zeilinger2011, Richter2012}. Nonlinear MPC is currently used on systems that evolve slowly over time, but there is active work to speed up computation \cite{Diehl2002, Schildbach2016,Diehl2009, Neunert2016}. Other methods skirt the issue of solving for optimal trajectories online by employing motion primitives \cite{Gillula2010, Dey2016}, or generating and choosing random trajectories at waypoints \cite{Kalakrishnan2011, Schwesinger2013}. 

There are several areas of research that add robustness offline through precomputation.  One method uses HJ analysis to guarantee tracking error bounds of a system with external disturbances \cite{Bansal2017}. A similar new approach, based on contraction theory and convex optimization, allows computation of offline error bounds that can then define safe tubes around a nominal dynamic trajectory computable online \cite{Singh2017}.  Another method uses motion primitives that are expanded by safety funnels \cite{Majumdar2016}.

Finally, some online control techniques can be applied add robustness to motion planning. Both linear and nonlinear MPC can add constraints to satisfy safety \cite{Richards2006, DiCairano2016,Hoy2015} by sacrificing speed of computation and/or maneuverability. For control-affine systems in which a control barrier function can be identified, a state-depented affine constraint can be used to ensure safety in an online optimization problem \cite{Ames2014}.

FaSTrack, the work presented in this paper, differs from the robust planning methods above because FaSTrack is designed to be modular and easy to use in conjunction with any path or trajectory planner. Additionally, FaSTrack can handle bounded external disturbances (e.g. wind) and work with both known and unknown environments with static obstacles.