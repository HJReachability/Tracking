% !TEX root = tracking.tex
\section{10D Quadrotor RRT Example \label{sec:results}}
\begin{figure*}
  \centering
  \begin{subfigure}{0.65\textwidth}
	  \includegraphics[width=\columnwidth,trim={0 0 0 0},clip]{fig/quad10D_example_cost}
	  \caption{Left: 2D slices of the 10D cost and tracking error functions in relative state space; contour lines show three level sets. Right: 3D projections of level sets at the same slice $(v_{x},v_{y},v_{z})=[1, -1, 1]$ m/s, $(\theta_{x},\omega_{x},\theta_{y},\omega_{y})=0$. Solid boxes show initial relative states, and transparent boxes show the corresponding tracking error bound. In practice we set the initial relative states to 0 to find the smallest invariant tracking error bound.\label{fig:quad10D_example}}
  \end{subfigure}~
  \begin{subfigure}{0.3\textwidth}
    \includegraphics[width=\columnwidth]{fig/quad10D_slices}
    \caption{Various 3D slices of the 10D relative states (solid) and the corresponding tracking error bound (transparent). \label{fig:quad10D_example_slices}}
  \end{subfigure} 
  \caption{Illustration of the tracking error function.}
\end{figure*} 

We demonstrate this framework with a 10D near-hover quadrotor developed in \cite{Bouffard12} tracking a 3D point source path generated by an RRT planner \cite{Kuffner2000},\cite{Kavraki1996}. First we perform the offline computations to acquire the tracking error bound and safety controller look-up tables. Next we set up a method to convert RRT paths to simple 3D trajectories. Finally we implement the online framework to navigate the 10D system through a 3D environment with static obstacles.

\subsection{Precomputation of 10D-3D system}
First we define the 10D dynamics of the tracking quadrotor and the 3D dynamics of a holonomic vehicle:
\begin{equation}
\small
\label{eq:Quad10D_dyn}
\begin{aligned}
\begin{array}{c}
\left[
\begin{array}{c}
\dot{x}\\
\dot{v_x}\\
\dot{\theta_x}\\
\dot\omega_x\\
\dot{y}\\
\dot{v_y}\\
\dot{\theta_y}\\
\dot\omega_y\\
\dot{z}\\
\dot{v_z}
\end{array}
\right]
=
\left[
\begin{array}{c}
v_x + d_x\\
g \tan \theta_x\\
-d_1 \theta_x + \omega_x\\
-d_0 \theta_x + n_0 a_x\\
v_y + d_y\\
g \tan \theta_y\\
-d_1 \theta_y + \omega_y\\
-d_0 \theta_y + n_0 a_y\\
v_z + d_z\\
k_T a_z - g
\end{array}
\right],\;\;
\left[
\begin{array}{c}
\dot{x}\\
\dot{y}\\
\dot{z}\\
\end{array}
\right] 
=
\left[
\begin{array}{c}
b_x\\
b_y\\
b_z \\
\end{array}
\right],
\end{array}\\
\end{aligned}
\end{equation}
where $(x, y, z)$ denote the position, $(v_x, v_y, v_z)$ velocity, $(\theta_x, \theta_y)$ the pitch and roll, and $(\omega_x, \omega_y)$ the pitch and roll rates. The controls of the 10D system are $(a_x, a_y, a_z)$, where $a_x$ and $a_y$ represent the desired pitch and roll angle ($|a_x|,|a_y|\leq10$ degrees), and $a_z$ represents the vertical thrust ($0\leq a_z\leq 1.5\,g$). The 3D system controls are $(b_x, b_y, b_z)$, and represent the velocity in each positional dimension ($|b_x|,|b_y|,|b_z|\leq0.5$ m/s). The disturbances in the 10D system $(d_x, d_y, d_z)$ are caused by wind, which acts on the velocity in each dimension ($|d_x|,|d_y|,|d_z|\leq0.1$ m/s). The values used for parameters $d_0,d_1,n_0,k_T,g$ were: $d_0=10,d_1=8,n_0=10,k_T=0.91,g=9.81$ m/s$^2$.

The relative dynamics between the two systems is defined using (\ref{eq:rdyn}). The states of the 3D dynamics are a subset of the 10D state space; the matrix Q used in the online computation matches the position states of both systems. Next we follow the setup in section \ref{sec:precomp} to create a cost function, which we then evaluate using HJ reachability until convergence to produce the invariant value function as in (\ref{eq:valfunc}). Using new methods in \cite{Chen2016DecouplingExact, Chen2016DecouplingJournal} we can decompose this system into 3 subsystems if the cost function is the 1-norm of the relative position. This cost function as well as the resulting value function can be seen projected onto the $x,y$ dimensions in Fig. \ref{fig:quad10D_example}.

Fig. \ref{fig:quad10D_example} also shows 3D positional projections of the mapping between initial relative state to maximum potential relative distance over all time (i.e. tracking error bound). If the real system starts exactly at the origin in relative coordinates, its tracking error bound will be a box of $\underline\valfunc = 0.81$ m in each direction. Slices of the 3D set and corresponding tracking error bounds are also shown in Fig. \ref{fig:quad10D_example_slices}. We save the look-up tables of the value function (i.e. the tracking error function) and its spatial gradients (i.e. the safety controller function).

\subsection{Online Planning with RRT and Sensing}
Our precomputed value function can serve as a tracking error bound, and its gradients form a look-up table for the optimal tracking controller. These can be combined with any planning algorithm such as MPC, RRT, or neural-network-based planners in a modular way. We used a simple multi-tree RRT planner implemented in MATLAB, modified from \cite{Gavin2013}. We assigned a speed of $0.5$ m/s to the piecewise linear paths obtained from the RRT planner, so that the planning model is as given in \eqref{eq:Quad10D_dyn}. Besides planning a path to the goal, the quadrotor must also sense obstacles in the vicinity. For illustration, we chose a simple virtual sensor that reveals obstacles within a range of 2 m in the $x$, $y$, or $z$ directions.

Once an obstacle is sensed, the RRT planner replans while taking into account all obstacles that have been sensed so far. To ensure that the quadrotor does not collide with the obstacles despite error in tracking, planning is done with respect to augmented obstacles that are ``expanded'' from the sensed obstacles by $\underline\valfunc$ in the $x$, $y$, and $z$ directions.

\begin{figure}
	\centering
	\begin{subfigure}[t]{0.49\columnwidth} \label{subfig:sim_4}
		\includegraphics[width=\columnwidth]{fig/1173}
	\end{subfigure}  
	\begin{subfigure}[t]{0.49\columnwidth} \label{subfig:sim_1}
		\includegraphics[width=\columnwidth]{fig/224}
		\caption{}
	\end{subfigure}
	
	\begin{subfigure}[t]{0.49\columnwidth} \label{subfig:sim_2}
		\includegraphics[width=\columnwidth]{fig/763}
		\caption{}
	\end{subfigure}  
	\begin{subfigure}[t]{0.49\columnwidth} \label{subfig:sim_3}
		\includegraphics[width=\columnwidth]{fig/1042}
		\caption{}
	\end{subfigure}
	\caption{Numerical simulation. $\tstate(t)$ is in blue, $\pstate(t)$ in magenta, unseen obstacles in gray, and seen obstacles in red. The translucent blue box represents $\TEB_\pstate(\tstate)$. Top left: the entire trajectory. Other subplots: close-up of the positions marked in top left subplot. The camera angle is adjusted for clarity of illustration. A video of this simulation can be found at 
		{\tt\href{https://youtu.be/ZVvyeK-a62E}{\nolinkurl{https://youtu.be/ZVvyeK-a62E}}}.
		\label{fig:sim}}
\end{figure}

On an unoptimized MATLAB implementation on a desktop computer with a Core i7-2600K CPU, each iteration took approximately $25$ ms on average. Most of this time is spent on planning: obtaining the tracking controller took approximately $5$ ms per iteration on average. The frequency of control was once every $100$ ms.

Fig. \ref{fig:sim} shows the simulation results. Four time snapshots are shown. The initial position is $(-12, 0, 0)$, and the goal position is $(12, 0, 0)$. The top left subplot shows the entire trajectory from beginning to end. In all plots, a magenta star represents the position of the planning model; its movement is based on the paths planned by RRT. The blue box around the magenta star represents the tracking error bound.

The position of the tracking model is shown in blue. Throughout the simulation, the tracking model's position is always inside the tracking error, in agreement with Proposition \ref{prop:main}. In addition, the tracking error bound never intersects with the obstacles, a consequence of the RRT planner planning with respect to a set of augmented obstacles (not shown). In the latter two subplots, one can see that the quadrotor appears to be exploring the environment briefly before reaching the goal. We did not employ any exploration algorithm; this exploration behavior emerges from replanning using RRT whenever a $3$ m$^2$ portion of an obstacle is sensed.