% !TEX root = tracking.tex
\section{General Framework \label{sec:framework}}
The online real-time framework of FaSTrack is shown in Fig. \ref{fig:fw_online}. At the center is the path or trajectory planner; our framework is agnostic to the planner, so any may be used. We will present an example using an RRT planner in Section \ref{sec:results}.
\begin{figure}[]
  \centering
	\includegraphics[width=1\columnwidth]{fig/framework_online}
	\caption{Online framework}
	\label{fig:fw_online}
	\vspace{-.1in}
\end{figure}

The first step of the online framework is to sense obstacles in the environment, and then augment the sensed obstacles by the precomputed $\TEB$ as described in Section \ref{sec:precomp}. This $\TEB$ is a safety margin that guarantees robustness despite the worst-case disturbance. Augmenting the obstacles by this margin can be thought of as equivalent to wrapping the planning system with a ``safety bubble". These augmented obstacles are given as inputs to the planner along with the current state of the planning system. The planner then outputs the next desired state of the planning system. 

The tracking system is a model of the physical system (such as a quadrotor). The hybrid tracking controller block takes in the state of the tracking system as well as the desired state of the planning system. Based on the relative state between these two systems, the hybrid tracking controller outputs a control signal to the tracking system. The hybrid tracking controller allows any general performance controller to be used when the tracking system is far from the tracking error violation. As the system approaches $\TEB$, the safety controller is employed.  The safety controller consists of a function (or look-up table) computed offline via HJ reachability, and guarantees that the error bound is not violated, \textit{despite the worst-case disturbance}. In addition, the table look-up operation is computationally inexpensive.

To determine both $\TEB$ and safety controller functions/look-up tables, an offline framework is used as shown in Fig. \ref{fig:fw_offline}. The planning and tracking system dynamics are plugged into an HJ reachability computation, which computes a value function that acts as the tracking error function/look-up table. The spatial gradients of the value function comprise the safety controller function/look-up table. These functions are independent of the online computations---they depend only on the \textit{relative} states and dynamics between the planning and tracking systems, not on the absolute states along the trajectory at execution time.

In the following sections we will first explain the precomputation steps in the offline framework. We will then explain the online framework and provide a complete example.
\begin{figure}[]
  \centering
	\includegraphics[width=0.9\columnwidth]{fig/framework_offline}
	\caption{Offline framework}
	\label{fig:fw_offline}
	%\vspace{-.1in}
\end{figure}

%Besides the virtual state, the planner also takes into account any obstacles that must be avoided. In order to be robust to disturbances, planning must be done with a safety margin that accounts for disturbances. A safety margin that guarantees robustness despite the worst-case disturbance is given by a tracking error bound obtained in the offline HJ reachability computation, shown in Fig. \ref{fig:fw_offline} and explained in detail in \MCnote{Section \ref{}}. The virtual and real system dynamics are used to compute a value function, which simultaneously gives the tracking error bound and the safety controller look-up table used by the hybrid tracking controller.
%
%\textbf{Maybe put next paragraph in the introduction}
%
%There are many fast planners that could potentially do planning in real-time; however, these typically cannot account for disturbances in a provably safe way. In addition, complex system models with nonlinear dynamics complicate planning algorithms (non-convex for MPC, more difficult for RRT). On the other hand, HJ reachability is able to handle disturbances, and is agnostic to system dynamics. In addition, provably guarantees can be provided. However, HJ reachability and in general formal verification methods can be very expensive to compute.
%
%Refer to figure: planning level and safety level. 
%
%In the safety level, we start with the error dynamics, and we compute two things: bubble which is fed into planner to plan with extra margin, and error-feedback controller for real-time control. These two can be computed offline independent of the planned path.
%
%In the planning level, any planning method such as MPC, RRT, etc. (cite some things) can be used. The planning level does not need to take into account disturbances, and can use simple system dynamics or even no dynamics at all. In fact we will be using a simple RRT planner which simply provides paths, in the form of a sequence of line segments, which are not dynamically feasible. 